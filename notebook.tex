
% Default to the notebook output style

    


% Inherit from the specified cell style.




    
\documentclass[11pt]{article}

    
    
    \usepackage[T1]{fontenc}
    % Nicer default font (+ math font) than Computer Modern for most use cases
    \usepackage{mathpazo}

    % Basic figure setup, for now with no caption control since it's done
    % automatically by Pandoc (which extracts ![](path) syntax from Markdown).
    \usepackage{graphicx}
    % We will generate all images so they have a width \maxwidth. This means
    % that they will get their normal width if they fit onto the page, but
    % are scaled down if they would overflow the margins.
    \makeatletter
    \def\maxwidth{\ifdim\Gin@nat@width>\linewidth\linewidth
    \else\Gin@nat@width\fi}
    \makeatother
    \let\Oldincludegraphics\includegraphics
    % Set max figure width to be 80% of text width, for now hardcoded.
    \renewcommand{\includegraphics}[1]{\Oldincludegraphics[width=.8\maxwidth]{#1}}
    % Ensure that by default, figures have no caption (until we provide a
    % proper Figure object with a Caption API and a way to capture that
    % in the conversion process - todo).
    \usepackage{caption}
    \DeclareCaptionLabelFormat{nolabel}{}
    \captionsetup{labelformat=nolabel}

    \usepackage{adjustbox} % Used to constrain images to a maximum size 
    \usepackage{xcolor} % Allow colors to be defined
    \usepackage{enumerate} % Needed for markdown enumerations to work
    \usepackage{geometry} % Used to adjust the document margins
    \usepackage{amsmath} % Equations
    \usepackage{amssymb} % Equations
    \usepackage{textcomp} % defines textquotesingle
    % Hack from http://tex.stackexchange.com/a/47451/13684:
    \AtBeginDocument{%
        \def\PYZsq{\textquotesingle}% Upright quotes in Pygmentized code
    }
    \usepackage{upquote} % Upright quotes for verbatim code
    \usepackage{eurosym} % defines \euro
    \usepackage[mathletters]{ucs} % Extended unicode (utf-8) support
    \usepackage[utf8x]{inputenc} % Allow utf-8 characters in the tex document
    \usepackage{fancyvrb} % verbatim replacement that allows latex
    \usepackage{grffile} % extends the file name processing of package graphics 
                         % to support a larger range 
    % The hyperref package gives us a pdf with properly built
    % internal navigation ('pdf bookmarks' for the table of contents,
    % internal cross-reference links, web links for URLs, etc.)
    \usepackage{hyperref}
    \usepackage{longtable} % longtable support required by pandoc >1.10
    \usepackage{booktabs}  % table support for pandoc > 1.12.2
    \usepackage[inline]{enumitem} % IRkernel/repr support (it uses the enumerate* environment)
    \usepackage[normalem]{ulem} % ulem is needed to support strikethroughs (\sout)
                                % normalem makes italics be italics, not underlines
    

    
    
    % Colors for the hyperref package
    \definecolor{urlcolor}{rgb}{0,.145,.698}
    \definecolor{linkcolor}{rgb}{.71,0.21,0.01}
    \definecolor{citecolor}{rgb}{.12,.54,.11}

    % ANSI colors
    \definecolor{ansi-black}{HTML}{3E424D}
    \definecolor{ansi-black-intense}{HTML}{282C36}
    \definecolor{ansi-red}{HTML}{E75C58}
    \definecolor{ansi-red-intense}{HTML}{B22B31}
    \definecolor{ansi-green}{HTML}{00A250}
    \definecolor{ansi-green-intense}{HTML}{007427}
    \definecolor{ansi-yellow}{HTML}{DDB62B}
    \definecolor{ansi-yellow-intense}{HTML}{B27D12}
    \definecolor{ansi-blue}{HTML}{208FFB}
    \definecolor{ansi-blue-intense}{HTML}{0065CA}
    \definecolor{ansi-magenta}{HTML}{D160C4}
    \definecolor{ansi-magenta-intense}{HTML}{A03196}
    \definecolor{ansi-cyan}{HTML}{60C6C8}
    \definecolor{ansi-cyan-intense}{HTML}{258F8F}
    \definecolor{ansi-white}{HTML}{C5C1B4}
    \definecolor{ansi-white-intense}{HTML}{A1A6B2}

    % commands and environments needed by pandoc snippets
    % extracted from the output of `pandoc -s`
    \providecommand{\tightlist}{%
      \setlength{\itemsep}{0pt}\setlength{\parskip}{0pt}}
    \DefineVerbatimEnvironment{Highlighting}{Verbatim}{commandchars=\\\{\}}
    % Add ',fontsize=\small' for more characters per line
    \newenvironment{Shaded}{}{}
    \newcommand{\KeywordTok}[1]{\textcolor[rgb]{0.00,0.44,0.13}{\textbf{{#1}}}}
    \newcommand{\DataTypeTok}[1]{\textcolor[rgb]{0.56,0.13,0.00}{{#1}}}
    \newcommand{\DecValTok}[1]{\textcolor[rgb]{0.25,0.63,0.44}{{#1}}}
    \newcommand{\BaseNTok}[1]{\textcolor[rgb]{0.25,0.63,0.44}{{#1}}}
    \newcommand{\FloatTok}[1]{\textcolor[rgb]{0.25,0.63,0.44}{{#1}}}
    \newcommand{\CharTok}[1]{\textcolor[rgb]{0.25,0.44,0.63}{{#1}}}
    \newcommand{\StringTok}[1]{\textcolor[rgb]{0.25,0.44,0.63}{{#1}}}
    \newcommand{\CommentTok}[1]{\textcolor[rgb]{0.38,0.63,0.69}{\textit{{#1}}}}
    \newcommand{\OtherTok}[1]{\textcolor[rgb]{0.00,0.44,0.13}{{#1}}}
    \newcommand{\AlertTok}[1]{\textcolor[rgb]{1.00,0.00,0.00}{\textbf{{#1}}}}
    \newcommand{\FunctionTok}[1]{\textcolor[rgb]{0.02,0.16,0.49}{{#1}}}
    \newcommand{\RegionMarkerTok}[1]{{#1}}
    \newcommand{\ErrorTok}[1]{\textcolor[rgb]{1.00,0.00,0.00}{\textbf{{#1}}}}
    \newcommand{\NormalTok}[1]{{#1}}
    
    % Additional commands for more recent versions of Pandoc
    \newcommand{\ConstantTok}[1]{\textcolor[rgb]{0.53,0.00,0.00}{{#1}}}
    \newcommand{\SpecialCharTok}[1]{\textcolor[rgb]{0.25,0.44,0.63}{{#1}}}
    \newcommand{\VerbatimStringTok}[1]{\textcolor[rgb]{0.25,0.44,0.63}{{#1}}}
    \newcommand{\SpecialStringTok}[1]{\textcolor[rgb]{0.73,0.40,0.53}{{#1}}}
    \newcommand{\ImportTok}[1]{{#1}}
    \newcommand{\DocumentationTok}[1]{\textcolor[rgb]{0.73,0.13,0.13}{\textit{{#1}}}}
    \newcommand{\AnnotationTok}[1]{\textcolor[rgb]{0.38,0.63,0.69}{\textbf{\textit{{#1}}}}}
    \newcommand{\CommentVarTok}[1]{\textcolor[rgb]{0.38,0.63,0.69}{\textbf{\textit{{#1}}}}}
    \newcommand{\VariableTok}[1]{\textcolor[rgb]{0.10,0.09,0.49}{{#1}}}
    \newcommand{\ControlFlowTok}[1]{\textcolor[rgb]{0.00,0.44,0.13}{\textbf{{#1}}}}
    \newcommand{\OperatorTok}[1]{\textcolor[rgb]{0.40,0.40,0.40}{{#1}}}
    \newcommand{\BuiltInTok}[1]{{#1}}
    \newcommand{\ExtensionTok}[1]{{#1}}
    \newcommand{\PreprocessorTok}[1]{\textcolor[rgb]{0.74,0.48,0.00}{{#1}}}
    \newcommand{\AttributeTok}[1]{\textcolor[rgb]{0.49,0.56,0.16}{{#1}}}
    \newcommand{\InformationTok}[1]{\textcolor[rgb]{0.38,0.63,0.69}{\textbf{\textit{{#1}}}}}
    \newcommand{\WarningTok}[1]{\textcolor[rgb]{0.38,0.63,0.69}{\textbf{\textit{{#1}}}}}
    
    
    % Define a nice break command that doesn't care if a line doesn't already
    % exist.
    \def\br{\hspace*{\fill} \\* }
    % Math Jax compatability definitions
    \def\gt{>}
    \def\lt{<}
    % Document parameters
    \title{numpy-random}
    
    
    

    % Pygments definitions
    
\makeatletter
\def\PY@reset{\let\PY@it=\relax \let\PY@bf=\relax%
    \let\PY@ul=\relax \let\PY@tc=\relax%
    \let\PY@bc=\relax \let\PY@ff=\relax}
\def\PY@tok#1{\csname PY@tok@#1\endcsname}
\def\PY@toks#1+{\ifx\relax#1\empty\else%
    \PY@tok{#1}\expandafter\PY@toks\fi}
\def\PY@do#1{\PY@bc{\PY@tc{\PY@ul{%
    \PY@it{\PY@bf{\PY@ff{#1}}}}}}}
\def\PY#1#2{\PY@reset\PY@toks#1+\relax+\PY@do{#2}}

\expandafter\def\csname PY@tok@w\endcsname{\def\PY@tc##1{\textcolor[rgb]{0.73,0.73,0.73}{##1}}}
\expandafter\def\csname PY@tok@c\endcsname{\let\PY@it=\textit\def\PY@tc##1{\textcolor[rgb]{0.25,0.50,0.50}{##1}}}
\expandafter\def\csname PY@tok@cp\endcsname{\def\PY@tc##1{\textcolor[rgb]{0.74,0.48,0.00}{##1}}}
\expandafter\def\csname PY@tok@k\endcsname{\let\PY@bf=\textbf\def\PY@tc##1{\textcolor[rgb]{0.00,0.50,0.00}{##1}}}
\expandafter\def\csname PY@tok@kp\endcsname{\def\PY@tc##1{\textcolor[rgb]{0.00,0.50,0.00}{##1}}}
\expandafter\def\csname PY@tok@kt\endcsname{\def\PY@tc##1{\textcolor[rgb]{0.69,0.00,0.25}{##1}}}
\expandafter\def\csname PY@tok@o\endcsname{\def\PY@tc##1{\textcolor[rgb]{0.40,0.40,0.40}{##1}}}
\expandafter\def\csname PY@tok@ow\endcsname{\let\PY@bf=\textbf\def\PY@tc##1{\textcolor[rgb]{0.67,0.13,1.00}{##1}}}
\expandafter\def\csname PY@tok@nb\endcsname{\def\PY@tc##1{\textcolor[rgb]{0.00,0.50,0.00}{##1}}}
\expandafter\def\csname PY@tok@nf\endcsname{\def\PY@tc##1{\textcolor[rgb]{0.00,0.00,1.00}{##1}}}
\expandafter\def\csname PY@tok@nc\endcsname{\let\PY@bf=\textbf\def\PY@tc##1{\textcolor[rgb]{0.00,0.00,1.00}{##1}}}
\expandafter\def\csname PY@tok@nn\endcsname{\let\PY@bf=\textbf\def\PY@tc##1{\textcolor[rgb]{0.00,0.00,1.00}{##1}}}
\expandafter\def\csname PY@tok@ne\endcsname{\let\PY@bf=\textbf\def\PY@tc##1{\textcolor[rgb]{0.82,0.25,0.23}{##1}}}
\expandafter\def\csname PY@tok@nv\endcsname{\def\PY@tc##1{\textcolor[rgb]{0.10,0.09,0.49}{##1}}}
\expandafter\def\csname PY@tok@no\endcsname{\def\PY@tc##1{\textcolor[rgb]{0.53,0.00,0.00}{##1}}}
\expandafter\def\csname PY@tok@nl\endcsname{\def\PY@tc##1{\textcolor[rgb]{0.63,0.63,0.00}{##1}}}
\expandafter\def\csname PY@tok@ni\endcsname{\let\PY@bf=\textbf\def\PY@tc##1{\textcolor[rgb]{0.60,0.60,0.60}{##1}}}
\expandafter\def\csname PY@tok@na\endcsname{\def\PY@tc##1{\textcolor[rgb]{0.49,0.56,0.16}{##1}}}
\expandafter\def\csname PY@tok@nt\endcsname{\let\PY@bf=\textbf\def\PY@tc##1{\textcolor[rgb]{0.00,0.50,0.00}{##1}}}
\expandafter\def\csname PY@tok@nd\endcsname{\def\PY@tc##1{\textcolor[rgb]{0.67,0.13,1.00}{##1}}}
\expandafter\def\csname PY@tok@s\endcsname{\def\PY@tc##1{\textcolor[rgb]{0.73,0.13,0.13}{##1}}}
\expandafter\def\csname PY@tok@sd\endcsname{\let\PY@it=\textit\def\PY@tc##1{\textcolor[rgb]{0.73,0.13,0.13}{##1}}}
\expandafter\def\csname PY@tok@si\endcsname{\let\PY@bf=\textbf\def\PY@tc##1{\textcolor[rgb]{0.73,0.40,0.53}{##1}}}
\expandafter\def\csname PY@tok@se\endcsname{\let\PY@bf=\textbf\def\PY@tc##1{\textcolor[rgb]{0.73,0.40,0.13}{##1}}}
\expandafter\def\csname PY@tok@sr\endcsname{\def\PY@tc##1{\textcolor[rgb]{0.73,0.40,0.53}{##1}}}
\expandafter\def\csname PY@tok@ss\endcsname{\def\PY@tc##1{\textcolor[rgb]{0.10,0.09,0.49}{##1}}}
\expandafter\def\csname PY@tok@sx\endcsname{\def\PY@tc##1{\textcolor[rgb]{0.00,0.50,0.00}{##1}}}
\expandafter\def\csname PY@tok@m\endcsname{\def\PY@tc##1{\textcolor[rgb]{0.40,0.40,0.40}{##1}}}
\expandafter\def\csname PY@tok@gh\endcsname{\let\PY@bf=\textbf\def\PY@tc##1{\textcolor[rgb]{0.00,0.00,0.50}{##1}}}
\expandafter\def\csname PY@tok@gu\endcsname{\let\PY@bf=\textbf\def\PY@tc##1{\textcolor[rgb]{0.50,0.00,0.50}{##1}}}
\expandafter\def\csname PY@tok@gd\endcsname{\def\PY@tc##1{\textcolor[rgb]{0.63,0.00,0.00}{##1}}}
\expandafter\def\csname PY@tok@gi\endcsname{\def\PY@tc##1{\textcolor[rgb]{0.00,0.63,0.00}{##1}}}
\expandafter\def\csname PY@tok@gr\endcsname{\def\PY@tc##1{\textcolor[rgb]{1.00,0.00,0.00}{##1}}}
\expandafter\def\csname PY@tok@ge\endcsname{\let\PY@it=\textit}
\expandafter\def\csname PY@tok@gs\endcsname{\let\PY@bf=\textbf}
\expandafter\def\csname PY@tok@gp\endcsname{\let\PY@bf=\textbf\def\PY@tc##1{\textcolor[rgb]{0.00,0.00,0.50}{##1}}}
\expandafter\def\csname PY@tok@go\endcsname{\def\PY@tc##1{\textcolor[rgb]{0.53,0.53,0.53}{##1}}}
\expandafter\def\csname PY@tok@gt\endcsname{\def\PY@tc##1{\textcolor[rgb]{0.00,0.27,0.87}{##1}}}
\expandafter\def\csname PY@tok@err\endcsname{\def\PY@bc##1{\setlength{\fboxsep}{0pt}\fcolorbox[rgb]{1.00,0.00,0.00}{1,1,1}{\strut ##1}}}
\expandafter\def\csname PY@tok@kc\endcsname{\let\PY@bf=\textbf\def\PY@tc##1{\textcolor[rgb]{0.00,0.50,0.00}{##1}}}
\expandafter\def\csname PY@tok@kd\endcsname{\let\PY@bf=\textbf\def\PY@tc##1{\textcolor[rgb]{0.00,0.50,0.00}{##1}}}
\expandafter\def\csname PY@tok@kn\endcsname{\let\PY@bf=\textbf\def\PY@tc##1{\textcolor[rgb]{0.00,0.50,0.00}{##1}}}
\expandafter\def\csname PY@tok@kr\endcsname{\let\PY@bf=\textbf\def\PY@tc##1{\textcolor[rgb]{0.00,0.50,0.00}{##1}}}
\expandafter\def\csname PY@tok@bp\endcsname{\def\PY@tc##1{\textcolor[rgb]{0.00,0.50,0.00}{##1}}}
\expandafter\def\csname PY@tok@fm\endcsname{\def\PY@tc##1{\textcolor[rgb]{0.00,0.00,1.00}{##1}}}
\expandafter\def\csname PY@tok@vc\endcsname{\def\PY@tc##1{\textcolor[rgb]{0.10,0.09,0.49}{##1}}}
\expandafter\def\csname PY@tok@vg\endcsname{\def\PY@tc##1{\textcolor[rgb]{0.10,0.09,0.49}{##1}}}
\expandafter\def\csname PY@tok@vi\endcsname{\def\PY@tc##1{\textcolor[rgb]{0.10,0.09,0.49}{##1}}}
\expandafter\def\csname PY@tok@vm\endcsname{\def\PY@tc##1{\textcolor[rgb]{0.10,0.09,0.49}{##1}}}
\expandafter\def\csname PY@tok@sa\endcsname{\def\PY@tc##1{\textcolor[rgb]{0.73,0.13,0.13}{##1}}}
\expandafter\def\csname PY@tok@sb\endcsname{\def\PY@tc##1{\textcolor[rgb]{0.73,0.13,0.13}{##1}}}
\expandafter\def\csname PY@tok@sc\endcsname{\def\PY@tc##1{\textcolor[rgb]{0.73,0.13,0.13}{##1}}}
\expandafter\def\csname PY@tok@dl\endcsname{\def\PY@tc##1{\textcolor[rgb]{0.73,0.13,0.13}{##1}}}
\expandafter\def\csname PY@tok@s2\endcsname{\def\PY@tc##1{\textcolor[rgb]{0.73,0.13,0.13}{##1}}}
\expandafter\def\csname PY@tok@sh\endcsname{\def\PY@tc##1{\textcolor[rgb]{0.73,0.13,0.13}{##1}}}
\expandafter\def\csname PY@tok@s1\endcsname{\def\PY@tc##1{\textcolor[rgb]{0.73,0.13,0.13}{##1}}}
\expandafter\def\csname PY@tok@mb\endcsname{\def\PY@tc##1{\textcolor[rgb]{0.40,0.40,0.40}{##1}}}
\expandafter\def\csname PY@tok@mf\endcsname{\def\PY@tc##1{\textcolor[rgb]{0.40,0.40,0.40}{##1}}}
\expandafter\def\csname PY@tok@mh\endcsname{\def\PY@tc##1{\textcolor[rgb]{0.40,0.40,0.40}{##1}}}
\expandafter\def\csname PY@tok@mi\endcsname{\def\PY@tc##1{\textcolor[rgb]{0.40,0.40,0.40}{##1}}}
\expandafter\def\csname PY@tok@il\endcsname{\def\PY@tc##1{\textcolor[rgb]{0.40,0.40,0.40}{##1}}}
\expandafter\def\csname PY@tok@mo\endcsname{\def\PY@tc##1{\textcolor[rgb]{0.40,0.40,0.40}{##1}}}
\expandafter\def\csname PY@tok@ch\endcsname{\let\PY@it=\textit\def\PY@tc##1{\textcolor[rgb]{0.25,0.50,0.50}{##1}}}
\expandafter\def\csname PY@tok@cm\endcsname{\let\PY@it=\textit\def\PY@tc##1{\textcolor[rgb]{0.25,0.50,0.50}{##1}}}
\expandafter\def\csname PY@tok@cpf\endcsname{\let\PY@it=\textit\def\PY@tc##1{\textcolor[rgb]{0.25,0.50,0.50}{##1}}}
\expandafter\def\csname PY@tok@c1\endcsname{\let\PY@it=\textit\def\PY@tc##1{\textcolor[rgb]{0.25,0.50,0.50}{##1}}}
\expandafter\def\csname PY@tok@cs\endcsname{\let\PY@it=\textit\def\PY@tc##1{\textcolor[rgb]{0.25,0.50,0.50}{##1}}}

\def\PYZbs{\char`\\}
\def\PYZus{\char`\_}
\def\PYZob{\char`\{}
\def\PYZcb{\char`\}}
\def\PYZca{\char`\^}
\def\PYZam{\char`\&}
\def\PYZlt{\char`\<}
\def\PYZgt{\char`\>}
\def\PYZsh{\char`\#}
\def\PYZpc{\char`\%}
\def\PYZdl{\char`\$}
\def\PYZhy{\char`\-}
\def\PYZsq{\char`\'}
\def\PYZdq{\char`\"}
\def\PYZti{\char`\~}
% for compatibility with earlier versions
\def\PYZat{@}
\def\PYZlb{[}
\def\PYZrb{]}
\makeatother


    % Exact colors from NB
    \definecolor{incolor}{rgb}{0.0, 0.0, 0.5}
    \definecolor{outcolor}{rgb}{0.545, 0.0, 0.0}



    
    % Prevent overflowing lines due to hard-to-break entities
    \sloppy 
    % Setup hyperref package
    \hypersetup{
      breaklinks=true,  % so long urls are correctly broken across lines
      colorlinks=true,
      urlcolor=urlcolor,
      linkcolor=linkcolor,
      citecolor=citecolor,
      }
    % Slightly bigger margins than the latex defaults
    
    \geometry{verbose,tmargin=1in,bmargin=1in,lmargin=1in,rmargin=1in}
    
    

    \begin{document}
    
    
    \maketitle
    
    

    
    \section{The numpy.random package}\label{the-numpy.random-package}

    \subsubsection{About Numpy.random}\label{about-numpy.random}

The numpy.random package is a sub-package of the Numpy
library\href{http://www.numpy.org/}{1} for the Python programming
language. Numpy is used for dealing with multi-dimensional arrays of
values and matrix operations in data analysis. While it's a powerful and
widely used package, users seldom interact directly with it - it's
typically accessed via one of the sub-packages such as numpy.random.

    \subsubsection{It's Purpose}\label{its-purpose}

The purpose of numpy.random is to provide users with a selection of
methods by which pseudo random numbers can be generated for use in
various applications. Perhaps the classic application is in the creation
of sampling plans.

While its not currently possible for a computer to create truly random
values, pseudo random numbers (derived from any method using
computational algorithms) offer a useful simulation of actual random
values. In theory, a higher degree of apparent randomness may
characterise numbers generated from methods using electromagnetic
atmospheric noise as a source input.
\href{https://en.wikipedia.org/wiki/Random_number_generation}{2} With
the advent of `quantum computing' perhaps it may finally become possible
to generate truly random values.

    \subsubsection{Who needs random
numbers?}\label{who-needs-random-numbers}

Aside from their frequent use in sampling and simulations, one
real-world application for pseudo random numbers which I have personally
encountered is in the online gaming industry. In this case, a stream of
pseudo random numbers is accessed by connected gaming terminals in
bookmakers shops, pubs etc to produce results in so-called ``games of
chance'', where the user's skills have no influence on the result. As a
mission critical component in the network, the random number generator
(RNG) runs 24x7 for 365 days a year, and any interruption of service is
flagged by loud alarms in the HQ where the operations team quickly work
to restore service!

    \subsubsection{Sampling Methods}\label{sampling-methods}

Sampling Methods can be classified into one of two
categories:\href{https://onlinecourses.science.psu.edu/stat100/node/18/}{3}

\begin{enumerate}
\def\labelenumi{\arabic{enumi})}
\item
  Probability Sampling: The sample has a known probability of being
  selected. All of the distributions illustrated here are examples of
  probability sampling.These techniques are more likely to yield a
  sample which is representative of the whole population, to within a
  calculated margin of error.
\item
  Non-probability Sampling: The sample does not have a known probability
  of being selected. In contrast with probability sampling, a
  non-probability sample is not a product of a randomized selection
  processes. Subjects in a non-probability sample are usually selected
  on the basis of their accessibility or by the personal judgment of the
  researcher. \href{https://explorable.com/non-probability-sampling}{4}
  Such techniques are unlikely to yield a sample which is representative
  of the whole population.
\end{enumerate}

    \subsubsection{Use of the Simple Random Data
functions}\label{use-of-the-simple-random-data-functions}

The ``Simple random data''
functions\href{https://docs.scipy.org/doc/numpy-1.14.1/reference/routines.random.html\#}{5}
in numpy.random provide a range of methods for creating random values in
a variety of arrays, formats and structures, depending on the specific
needs of the user. This is a good starting point for devising a simple
sampling plan.

    \subsubsection{Using numpy.random.rand}\label{using-numpy.random.rand}

Simple random data functions in numpy.random begin with
numpy.random.rand
\href{https://docs.scipy.org/doc/numpy-1.15.1/reference/generated/numpy.random.rand.html\#numpy.random.rand}{6},
which allows you to create an array of a given shape and populate it
with random floating values of a uniform distribution, from 0 to
0.99999999. For example, here's an array of 3 sets of sample values, of
4 rows and 3 columns each, generated by np.random.rand

    \begin{Verbatim}[commandchars=\\\{\}]
{\color{incolor}In [{\color{incolor}3}]:} \PY{k+kn}{import} \PY{n+nn}{numpy} \PY{k}{as} \PY{n+nn}{np}
\end{Verbatim}


    \begin{Verbatim}[commandchars=\\\{\}]
{\color{incolor}In [{\color{incolor}4}]:} \PY{n}{np}\PY{o}{.}\PY{n}{random}\PY{o}{.}\PY{n}{rand}\PY{p}{(}\PY{l+m+mi}{3}\PY{p}{,}\PY{l+m+mi}{4}\PY{p}{,}\PY{l+m+mi}{3}\PY{p}{)}
\end{Verbatim}


\begin{Verbatim}[commandchars=\\\{\}]
{\color{outcolor}Out[{\color{outcolor}4}]:} array([[[0.81893648, 0.35894255, 0.69974417],
                [0.2048686 , 0.10093235, 0.46919491],
                [0.1994188 , 0.5658538 , 0.1875186 ],
                [0.92495191, 0.79688265, 0.91042378]],
        
               [[0.76145577, 0.95218081, 0.99571217],
                [0.12313656, 0.93317708, 0.43235349],
                [0.76512726, 0.89835127, 0.22475713],
                [0.25918935, 0.37876375, 0.45609549]],
        
               [[0.9402218 , 0.87519641, 0.84522271],
                [0.08107309, 0.37347316, 0.50104148],
                [0.24723632, 0.49645582, 0.13205144],
                [0.66643175, 0.56962985, 0.89748852]]])
\end{Verbatim}
            
    Even in such small sample sizes, each of these values is almost as
likely to appear in the array as any other, as it's a uniform
distribution.

    \subsubsection{Using numpy.random.int}\label{using-numpy.random.int}

    Another simple random data function is numpy.random.int
\href{https://docs.scipy.org/doc/numpy-1.14.1/reference/generated/numpy.random.randint.html\#numpy.random.randint}{7},
which can create an array of integers between defined low \& high
limits, from the ``discrete uniform'' distribution in the closed
interval {[}low, high{]}. For example, to create an array of 10
integers, greater than 0 and less than or equal to 100 :

    \begin{Verbatim}[commandchars=\\\{\}]
{\color{incolor}In [{\color{incolor}14}]:} \PY{n}{np}\PY{o}{.}\PY{n}{random}\PY{o}{.}\PY{n}{randint}\PY{p}{(}\PY{l+m+mi}{100}\PY{p}{,} \PY{n}{size}\PY{o}{=}\PY{l+m+mi}{10}\PY{p}{)}
\end{Verbatim}


\begin{Verbatim}[commandchars=\\\{\}]
{\color{outcolor}Out[{\color{outcolor}14}]:} array([57,  2, 11, 72, 59, 26, 83, 54, 38, 82])
\end{Verbatim}
            
    \subsubsection{Permutations}\label{permutations}

The ``Permutation'' functions offer a range of methods for randomising
the arrangement (or order) of a defined sequence of values. This could
be a good starting point for a programmer creating a music shuffle
function.

    \begin{Verbatim}[commandchars=\\\{\}]
{\color{incolor}In [{\color{incolor}4}]:} \PY{k+kn}{import} \PY{n+nn}{numpy} \PY{k}{as} \PY{n+nn}{np}
\end{Verbatim}


    Here is an example of a permutation function using the shuffle command,
starting with an initial array:

    \begin{Verbatim}[commandchars=\\\{\}]
{\color{incolor}In [{\color{incolor}15}]:} \PY{n}{arr} \PY{o}{=} \PY{n}{np}\PY{o}{.}\PY{n}{arange}\PY{p}{(}\PY{l+m+mi}{15}\PY{p}{)}
         \PY{n}{arr}
\end{Verbatim}


\begin{Verbatim}[commandchars=\\\{\}]
{\color{outcolor}Out[{\color{outcolor}15}]:} array([ 0,  1,  2,  3,  4,  5,  6,  7,  8,  9, 10, 11, 12, 13, 14])
\end{Verbatim}
            
    Here it is, randomly shuffled:

    \begin{Verbatim}[commandchars=\\\{\}]
{\color{incolor}In [{\color{incolor}17}]:} \PY{n}{np}\PY{o}{.}\PY{n}{random}\PY{o}{.}\PY{n}{shuffle}\PY{p}{(}\PY{n}{arr}\PY{p}{)}
         \PY{n}{arr}
\end{Verbatim}


\begin{Verbatim}[commandchars=\\\{\}]
{\color{outcolor}Out[{\color{outcolor}17}]:} array([11,  4,  7,  1,  0,  8, 10, 13,  6, 14,  3, 12,  9,  5,  2])
\end{Verbatim}
            
    Incidently, how many permutations are possible in this example? How many
unique ways could you arrange 15 books on a shelf? The answer is
15\emph{14}13\emph{12}11\emph{10}9\emph{8}7\emph{6}5\emph{4}3\emph{2}1
or "factorial 15", expressed as 15! - and the number of permutations is
1.3076744e+12 .... or 1,307,674,368,000

    \subsection{About Probability
Distributions}\label{about-probability-distributions}

    A~probability distribution~is a mathematical function that provides the
probabilities of occurrence of different possible outcomes in
an~experiment.\href{https://en.wikipedia.org/wiki/Probability_distribution\#Applications}{8}
The concept of the probability distribution and the random variables
which they describe underlies the mathematical discipline of probability
theory, and the science of statistics. There is spread or variability in
almost any value that can be measured in a population (e.g. height of
people, durability of a metal, sales growth, traffic flow, etc.).

Examples of Probability distributions include the following:

    \subsubsection{Visualising
Distributions}\label{visualising-distributions}

To illustrate the distribution of random values generated by the
np.random.rand function, we can use it to generate a sample of values
and then plot them using matplotlib:

    \subsubsection{Uniform}\label{uniform}

    \begin{Verbatim}[commandchars=\\\{\}]
{\color{incolor}In [{\color{incolor}3}]:} \PY{k+kn}{import} \PY{n+nn}{matplotlib}\PY{n+nn}{.}\PY{n+nn}{pyplot} \PY{k}{as} \PY{n+nn}{plt}
\end{Verbatim}


    Generate 30,000 random values of between 0 and 0.99999999

    \begin{Verbatim}[commandchars=\\\{\}]
{\color{incolor}In [{\color{incolor}7}]:} \PY{n}{x} \PY{o}{=} \PY{n}{np}\PY{o}{.}\PY{n}{random}\PY{o}{.}\PY{n}{rand}\PY{p}{(}\PY{l+m+mi}{30000}\PY{p}{)}
        \PY{n}{x}
\end{Verbatim}


\begin{Verbatim}[commandchars=\\\{\}]
{\color{outcolor}Out[{\color{outcolor}7}]:} array([0.34202469, 0.84396772, 0.1611977 , {\ldots}, 0.30696916, 0.72532294,
               0.82767338])
\end{Verbatim}
            
    Plot the distribution of these values in a histogram

    \begin{Verbatim}[commandchars=\\\{\}]
{\color{incolor}In [{\color{incolor}8}]:} \PY{n}{plt}\PY{o}{.}\PY{n}{hist}\PY{p}{(}\PY{n}{x}\PY{p}{)}
\end{Verbatim}


\begin{Verbatim}[commandchars=\\\{\}]
{\color{outcolor}Out[{\color{outcolor}8}]:} (array([2948., 2998., 2955., 3063., 2998., 2940., 3051., 3101., 2879.,
                3067.]),
         array([6.15054398e-05, 1.00051879e-01, 2.00042253e-01, 3.00032627e-01,
                4.00023000e-01, 5.00013374e-01, 6.00003748e-01, 6.99994121e-01,
                7.99984495e-01, 8.99974869e-01, 9.99965242e-01]),
         <a list of 10 Patch objects>)
\end{Verbatim}
            
    \begin{center}
    \adjustimage{max size={0.9\linewidth}{0.9\paperheight}}{output_28_1.png}
    \end{center}
    { \hspace*{\fill} \\}
    
    That's what a randomly generated, essentially uniform distribution looks
like. It's uniform insofaras every value in the range defined had an
equal chance of being picked. We see approx 3,000 appearances of each
value.

Another function which generates a uniform distribution
isnp.random.uniform :

    \begin{Verbatim}[commandchars=\\\{\}]
{\color{incolor}In [{\color{incolor}11}]:} \PY{n}{s} \PY{o}{=} \PY{n}{np}\PY{o}{.}\PY{n}{random}\PY{o}{.}\PY{n}{uniform}\PY{p}{(}\PY{o}{\PYZhy{}}\PY{l+m+mi}{10}\PY{p}{,}\PY{o}{\PYZhy{}}\PY{l+m+mi}{9}\PY{p}{,}\PY{l+m+mi}{100}\PY{p}{)}
         \PY{n}{s}
\end{Verbatim}


\begin{Verbatim}[commandchars=\\\{\}]
{\color{outcolor}Out[{\color{outcolor}11}]:} array([-9.42966673, -9.43826041, -9.15772765, -9.02574039, -9.45295171,
                -9.13661873, -9.20970295, -9.08172512, -9.96541557, -9.87461046,
                -9.26951045, -9.78469684, -9.85440231, -9.08656857, -9.87382243,
                -9.83658855, -9.49320976, -9.56933105, -9.92882012, -9.10878303,
                -9.37818345, -9.26286676, -9.56605076, -9.57175948, -9.03579986,
                -9.56304116, -9.62518655, -9.49122994, -9.75068558, -9.13240214,
                -9.19873253, -9.07058949, -9.92578653, -9.89969589, -9.81898972,
                -9.91428485, -9.38656149, -9.23385309, -9.82110294, -9.69510141,
                -9.1176082 , -9.21323478, -9.09188739, -9.23245698, -9.76992772,
                -9.60325804, -9.1819301 , -9.86632107, -9.95424725, -9.15422782,
                -9.31136197, -9.78764175, -9.85983893, -9.24310035, -9.09782721,
                -9.56433275, -9.258888  , -9.915299  , -9.3963184 , -9.11897165,
                -9.92470502, -9.28982454, -9.47440205, -9.0750263 , -9.83972934,
                -9.85013245, -9.32456063, -9.83342988, -9.02059918, -9.87814024,
                -9.26739907, -9.47300001, -9.09665196, -9.51381173, -9.00956919,
                -9.73405016, -9.24045157, -9.43650531, -9.58332736, -9.24741101,
                -9.00946425, -9.25445408, -9.58370999, -9.5978992 , -9.07377975,
                -9.83406521, -9.99815617, -9.33983035, -9.15933814, -9.24325862,
                -9.48953442, -9.54631307, -9.08831483, -9.61774408, -9.55885809,
                -9.6364229 , -9.57442875, -9.37428849, -9.5438914 , -9.80666539])
\end{Verbatim}
            
    the np.random.uniform function seems to show a quite different looking
distribution for this 100 sample, but the larger the sample size the
more uniform it looks.

    \begin{Verbatim}[commandchars=\\\{\}]
{\color{incolor}In [{\color{incolor}12}]:} \PY{n}{plt}\PY{o}{.}\PY{n}{hist}\PY{p}{(}\PY{n}{s}\PY{p}{)}
\end{Verbatim}


\begin{Verbatim}[commandchars=\\\{\}]
{\color{outcolor}Out[{\color{outcolor}12}]:} (array([ 9., 14.,  5.,  5., 13.,  9.,  7., 14., 10., 14.]),
          array([-9.99815617, -9.89928698, -9.80041779, -9.70154859, -9.6026794 ,
                 -9.50381021, -9.40494102, -9.30607182, -9.20720263, -9.10833344,
                 -9.00946425]),
          <a list of 10 Patch objects>)
\end{Verbatim}
            
    \begin{center}
    \adjustimage{max size={0.9\linewidth}{0.9\paperheight}}{output_32_1.png}
    \end{center}
    { \hspace*{\fill} \\}
    
    \subsubsection{Normal}\label{normal}

    A Normal (or Gaussian) distribution is related to real-valued quantities
that grow linearly (e.g. errors, offsets). Its the most common
continuous distribution. We can use the 'numpy.random.normal' function
which generates a normal distribution looking more like the classic
'bell curve'.

    \begin{Verbatim}[commandchars=\\\{\}]
{\color{incolor}In [{\color{incolor}7}]:} \PY{n}{y} \PY{o}{=} \PY{n}{np}\PY{o}{.}\PY{n}{random}\PY{o}{.}\PY{n}{normal}\PY{p}{(}\PY{l+m+mi}{0}\PY{p}{,}\PY{l+m+mf}{0.1}\PY{p}{,}\PY{l+m+mi}{1000}\PY{p}{)}
\end{Verbatim}


    \begin{Verbatim}[commandchars=\\\{\}]
{\color{incolor}In [{\color{incolor}8}]:} \PY{n}{plt}\PY{o}{.}\PY{n}{hist}\PY{p}{(}\PY{n}{y}\PY{p}{)}
\end{Verbatim}


\begin{Verbatim}[commandchars=\\\{\}]
{\color{outcolor}Out[{\color{outcolor}8}]:} (array([  5.,  32.,  86., 190., 242., 242., 132.,  54.,  14.,   3.]),
         array([-0.32178286, -0.25519221, -0.18860157, -0.12201092, -0.05542028,
                 0.01117037,  0.07776101,  0.14435166,  0.2109423 ,  0.27753295,
                 0.34412359]),
         <a list of 10 Patch objects>)
\end{Verbatim}
            
    \begin{center}
    \adjustimage{max size={0.9\linewidth}{0.9\paperheight}}{output_36_1.png}
    \end{center}
    { \hspace*{\fill} \\}
    
    This Normal distribution shows observations clustered around the Mean
value of zero. The chances of values close to zero being picked are
greater than for those at the extremities of the distribution. Typically
95\% of all observations in a Normal distribution fall within +/- 2
Standard Deviations from the Mean.

    Notice that its the 1st digit in np.random.normal (0,0.1,10000) which
defines the MEAN in this distribution - in this case 0. You can make it
any value you choose. Equally, the 2nd digit defines the intervals in
the range.

    \subsubsection{SUNDAY 11.20am - Now lets take a look at Random Number
Generation, using
numpy.random.RandomState}\label{sunday-11.20am---now-lets-take-a-look-at-random-number-generation-using-numpy.random.randomstate}

    \begin{Verbatim}[commandchars=\\\{\}]
{\color{incolor}In [{\color{incolor}9}]:} \PY{k+kn}{import} \PY{n+nn}{numpy} \PY{k}{as} \PY{n+nn}{np}
\end{Verbatim}


    \begin{Verbatim}[commandchars=\\\{\}]
{\color{incolor}In [{\color{incolor}10}]:} \PY{n}{z} \PY{o}{=} \PY{n}{np}\PY{o}{.}\PY{n}{random}\PY{o}{.}\PY{n}{RandomState}\PY{p}{(}\PY{p}{[}\PY{l+m+mi}{4}\PY{p}{]}\PY{p}{)}
         \PY{n}{z}
\end{Verbatim}


\begin{Verbatim}[commandchars=\\\{\}]
{\color{outcolor}Out[{\color{outcolor}10}]:} <mtrand.RandomState at 0x11a07cd80>
\end{Verbatim}
            
    \subsection{Bayes Theorem - a
Simulation}\label{bayes-theorem---a-simulation}

    \begin{Verbatim}[commandchars=\\\{\}]
{\color{incolor}In [{\color{incolor}11}]:} \PY{k+kn}{import} \PY{n+nn}{numpy} \PY{k}{as} \PY{n+nn}{np}
         \PY{n+nb}{print}\PY{p}{(}\PY{n}{np}\PY{o}{.}\PY{n}{random}\PY{o}{.}\PY{n}{binomial}\PY{p}{(}\PY{l+m+mi}{1}\PY{p}{,}\PY{l+m+mf}{0.01}\PY{p}{)}\PY{p}{)}
         \PY{c+c1}{\PYZsh{} experiment runs once(1) with 1\PYZpc{} prob of a positive result}
         \PY{c+c1}{\PYZsh{} expect result will be zero 99 times out of 100}
         
         \PY{n}{x} \PY{o}{=} \PY{n}{np}\PY{o}{.}\PY{n}{random}\PY{o}{.}\PY{n}{binomial}\PY{p}{(}\PY{l+m+mi}{1}\PY{p}{,} \PY{l+m+mf}{0.01}\PY{p}{,} \PY{l+m+mi}{1000}\PY{p}{)}
         \PY{n+nb}{print}\PY{p}{(}\PY{n}{np}\PY{o}{.}\PY{n}{sum}\PY{p}{(}\PY{n}{x}\PY{p}{)}\PY{p}{)}
         \PY{c+c1}{\PYZsh{} now it runs 1,000 times,how often is the result positive?}
         \PY{c+c1}{\PYZsh{} expect results to cluster around a mean of 10}
\end{Verbatim}


    \begin{Verbatim}[commandchars=\\\{\}]
0
12

    \end{Verbatim}

    \subsubsection{coding the simulation}\label{coding-the-simulation}

    \begin{Verbatim}[commandchars=\\\{\}]
{\color{incolor}In [{\color{incolor}12}]:} \PY{c+c1}{\PYZsh{} Helper function, returns True with probability P, False otherwise.\PYZbs{}n\PYZdq{},}
         \PY{k}{def} \PY{n+nf}{true\PYZus{}with\PYZus{}prob\PYZus{}p}\PY{p}{(}\PY{n}{p}\PY{p}{)}\PY{p}{:}
             \PY{k}{return} \PY{k+kc}{True} \PY{k}{if} \PY{n}{np}\PY{o}{.}\PY{n}{random}\PY{o}{.}\PY{n}{binomial}\PY{p}{(}\PY{l+m+mi}{1}\PY{p}{,} \PY{n}{p}\PY{p}{)} \PY{o}{==} \PY{l+m+mi}{1} \PY{k}{else} \PY{k+kc}{False}
             
             \PY{c+c1}{\PYZsh{} Simulate the selection of a random person from the population.}
             \PY{c+c1}{\PYZsh{} Return True if they are a drug user, False otherwise.}
             \PY{c+c1}{\PYZsh{} True is returned with probability 0.005.}
             
         \PY{k}{def} \PY{n+nf}{select\PYZus{}random\PYZus{}person}\PY{p}{(}\PY{p}{)}\PY{p}{:}
             \PY{k}{return} \PY{n}{true\PYZus{}with\PYZus{}prob\PYZus{}p}\PY{p}{(}\PY{l+m+mf}{0.005}\PY{p}{)}\PYZbs{}
             
             \PY{c+c1}{\PYZsh{} Simulate the testing of a person from the population.}
             \PY{c+c1}{\PYZsh{} Return True if they test positive, False otherwise.}
             \PY{c+c1}{\PYZsh{} Non\PYZhy{}users test positive with probability 0.01.}
             \PY{c+c1}{\PYZsh{} Users test positive with probability 0.99.}
         \PY{k}{def} \PY{n+nf}{test\PYZus{}person}\PY{p}{(}\PY{n}{user}\PY{p}{)}\PY{p}{:}
             \PY{k}{if} \PY{n}{user}\PY{p}{:}\PYZbs{}
                 \PY{k}{return} \PY{n}{true\PYZus{}with\PYZus{}prob\PYZus{}p}\PY{p}{(}\PY{l+m+mf}{0.99}\PY{p}{)}
             \PY{k}{else}\PY{p}{:}
                 \PY{k}{return} \PY{n}{true\PYZus{}with\PYZus{}prob\PYZus{}p}\PY{p}{(}\PY{l+m+mf}{0.01}\PY{p}{)}
             
             \PY{c+c1}{\PYZsh{} Run an experiment \PYZhy{} take a random person from the population}
             \PY{c+c1}{\PYZsh{} and test whether or not they are positive.}
         \PY{k}{def} \PY{n+nf}{run\PYZus{}experiment}\PY{p}{(}\PY{p}{)}\PY{p}{:}
             \PY{n}{user} \PY{o}{=} \PY{n}{select\PYZus{}random\PYZus{}person}\PY{p}{(}\PY{p}{)}
             \PY{n}{test} \PY{o}{=} \PY{n}{test\PYZus{}person}\PY{p}{(}\PY{n}{user}\PY{p}{)}
             \PY{k}{return} \PY{p}{(}\PY{n}{user}\PY{p}{,} \PY{n}{test}\PY{p}{)}
\end{Verbatim}


    \begin{Verbatim}[commandchars=\\\{\}]
{\color{incolor}In [{\color{incolor}13}]:} \PY{c+c1}{\PYZsh{} Run the experiment 10,000 times.}
         \PY{n}{y} \PY{o}{=} \PY{p}{[}\PY{n}{run\PYZus{}experiment}\PY{p}{(}\PY{p}{)} \PY{k}{for} \PY{n}{i} \PY{o+ow}{in} \PY{n+nb}{range}\PY{p}{(}\PY{l+m+mi}{10000}\PY{p}{)}\PY{p}{]}
         \PY{c+c1}{\PYZsh{} Count the number of users who tested positive.}
         \PY{n}{user\PYZus{}and\PYZus{}positive} \PY{o}{=} \PY{p}{[}\PY{k+kc}{True} \PY{k}{for} \PY{n}{i} \PY{o+ow}{in} \PY{n}{y} \PY{k}{if} \PY{n}{i}\PY{p}{[}\PY{l+m+mi}{0}\PY{p}{]} \PY{o}{==} \PY{k+kc}{True} \PY{o+ow}{and} \PY{n}{i}\PY{p}{[}\PY{l+m+mi}{1}\PY{p}{]} \PY{o}{==} \PY{k+kc}{True}\PY{p}{]}
         
         \PY{c+c1}{\PYZsh{} Count the number of non\PYZhy{}users who tested positive.}
         \PY{n}{nonuser\PYZus{}and\PYZus{}positive} \PY{o}{=} \PY{p}{[}\PY{k+kc}{True}  \PY{k}{for} \PY{n}{i} \PY{o+ow}{in} \PY{n}{y} \PY{k}{if} \PY{n}{i}\PY{p}{[}\PY{l+m+mi}{0}\PY{p}{]} \PY{o}{==} \PY{k+kc}{False} \PY{o+ow}{and} \PY{n}{i}\PY{p}{[}\PY{l+m+mi}{1}\PY{p}{]} \PY{o}{==} \PY{k+kc}{True}\PY{p}{]}
\end{Verbatim}


    \begin{Verbatim}[commandchars=\\\{\}]
{\color{incolor}In [{\color{incolor}14}]:} \PY{n}{np}\PY{o}{.}\PY{n}{sum}\PY{p}{(}\PY{n}{user\PYZus{}and\PYZus{}positive}\PY{p}{)}
\end{Verbatim}


\begin{Verbatim}[commandchars=\\\{\}]
{\color{outcolor}Out[{\color{outcolor}14}]:} 41
\end{Verbatim}
            
    \begin{Verbatim}[commandchars=\\\{\}]
{\color{incolor}In [{\color{incolor}15}]:} \PY{n}{np}\PY{o}{.}\PY{n}{sum}\PY{p}{(}\PY{n}{nonuser\PYZus{}and\PYZus{}positive}\PY{p}{)}
\end{Verbatim}


\begin{Verbatim}[commandchars=\\\{\}]
{\color{outcolor}Out[{\color{outcolor}15}]:} 92
\end{Verbatim}
            
    \begin{Verbatim}[commandchars=\\\{\}]
{\color{incolor}In [{\color{incolor}16}]:} \PY{k+kn}{import} \PY{n+nn}{matplotlib}\PY{n+nn}{.}\PY{n+nn}{pyplot} \PY{k}{as} \PY{n+nn}{plt}
         \PY{n}{plt}\PY{o}{.}\PY{n}{show}\PY{p}{(}\PY{p}{)}
         
         \PY{n}{plt}\PY{o}{.}\PY{n}{bar}\PY{p}{(}\PY{p}{[}\PY{l+m+mi}{0}\PY{p}{,} \PY{l+m+mi}{1}\PY{p}{]}\PY{p}{,} \PY{p}{[}\PY{n}{np}\PY{o}{.}\PY{n}{sum}\PY{p}{(}\PY{n}{user\PYZus{}and\PYZus{}positive}\PY{p}{)}\PY{p}{,} \PY{n}{np}\PY{o}{.}\PY{n}{sum}\PY{p}{(}\PY{n}{nonuser\PYZus{}and\PYZus{}positive}\PY{p}{)}\PY{p}{]}\PY{p}{)}
         \PY{n}{plt}\PY{o}{.}\PY{n}{xticks}\PY{p}{(}\PY{p}{[}\PY{l+m+mi}{0}\PY{p}{,} \PY{l+m+mi}{1}\PY{p}{]}\PY{p}{,} \PY{p}{(}\PY{l+s+s1}{\PYZsq{}}\PY{l+s+s1}{Users}\PY{l+s+s1}{\PYZsq{}}\PY{p}{,} \PY{p}{(}\PY{l+s+s1}{\PYZsq{}}\PY{l+s+s1}{Non\PYZhy{}Users}\PY{l+s+s1}{\PYZsq{}}\PY{p}{)}\PY{p}{)}\PY{p}{)}
         \PY{n}{plt}\PY{o}{.}\PY{n}{title}\PY{p}{(}\PY{l+s+s2}{\PYZdq{}}\PY{l+s+s2}{People who tested positive}\PY{l+s+s2}{\PYZdq{}}\PY{p}{)}
\end{Verbatim}


\begin{Verbatim}[commandchars=\\\{\}]
{\color{outcolor}Out[{\color{outcolor}16}]:} Text(0.5,1,'People who tested positive')
\end{Verbatim}
            
    \begin{center}
    \adjustimage{max size={0.9\linewidth}{0.9\paperheight}}{output_49_1.png}
    \end{center}
    { \hspace*{\fill} \\}
    
    \subsection{Analysis}\label{analysis}

\subsubsection{\texorpdfstring{P(User\Positive) = P(Positive\User) *
P(user)}{P(User) = P(Positive) * P(user)}}\label{puser-ppositive-puser}

\subsubsection{all / P(Positive)}\label{all-ppositive}

\subsubsection{\texorpdfstring{also = P(Positive\User) *
P(user)}{also = P(Positive) * P(user)}}\label{also-ppositive-puser}

\subsubsection{\texorpdfstring{all / P(Positive\User) * P(user) +
P(Positive\NonUser) *
P(NonUser)}{all / P(Positive) * P(user) + P(Positive) * P(NonUser)}}\label{all-ppositive-puser-ppositive-pnonuser}

    \begin{Verbatim}[commandchars=\\\{\}]
{\color{incolor}In [{\color{incolor}17}]:} \PY{c+c1}{\PYZsh{} Probability that you\PYZsq{}re a user.}
         \PY{n}{p\PYZus{}user} \PY{o}{=} \PY{l+m+mf}{0.005}
         
         \PY{c+c1}{\PYZsh{} Probability that you\PYZsq{}re a non\PYZhy{}user.}
         \PY{n}{p\PYZus{}nonuser} \PY{o}{=} \PY{l+m+mi}{1} \PY{o}{\PYZhy{}} \PY{n}{p\PYZus{}user}
         
         \PY{c+c1}{\PYZsh{} Probability that a user tests positive.}
         \PY{n}{p\PYZus{}positive\PYZus{}user} \PY{o}{=} \PY{l+m+mf}{0.99}
         
         \PY{c+c1}{\PYZsh{} Probability that a non\PYZhy{}user tests negative.}
         \PY{n}{p\PYZus{}positive\PYZus{}nonuser} \PY{o}{=} \PY{l+m+mf}{1.0} \PY{o}{\PYZhy{}} \PY{l+m+mf}{0.99}
         
         \PY{c+c1}{\PYZsh{} Probability that you test positive.}
         \PY{n}{p\PYZus{}positive} \PY{o}{=} \PY{n}{p\PYZus{}positive\PYZus{}user} \PY{o}{*} \PY{n}{p\PYZus{}user} \PY{o}{+} \PY{n}{p\PYZus{}positive\PYZus{}nonuser} \PY{o}{*} \PY{n}{p\PYZus{}nonuser}
         
         \PY{c+c1}{\PYZsh{} Bayes\PYZsq{} theorem.}
         \PY{n}{top\PYZus{}line} \PY{o}{=} \PY{n}{p\PYZus{}positive\PYZus{}user} \PY{o}{*} \PY{n}{p\PYZus{}user}
         \PY{n}{bottom\PYZus{}line} \PY{o}{=} \PY{n}{p\PYZus{}positive}
         \PY{n}{p\PYZus{}user\PYZus{}positive} \PY{o}{=} \PY{n}{top\PYZus{}line} \PY{o}{/} \PY{n}{bottom\PYZus{}line}
         
         \PY{c+c1}{\PYZsh{} Show result.}
         \PY{n+nb}{print}\PY{p}{(}\PY{n}{p\PYZus{}user\PYZus{}positive}\PY{p}{)}
\end{Verbatim}


    \begin{Verbatim}[commandchars=\\\{\}]
0.33221476510067094

    \end{Verbatim}

    \subsection{Exploring PANDAS}\label{exploring-pandas}

    \paragraph{About the Iris data set from UC Irvine's machine learning
repository}\label{about-the-iris-data-set-from-uc-irvines-machine-learning-repository}

(https://archive.ics.uci.edu/ml/datasets/iris)

    \subsection{Loading Data}\label{loading-data}

    \begin{Verbatim}[commandchars=\\\{\}]
{\color{incolor}In [{\color{incolor}18}]:} \PY{k+kn}{import} \PY{n+nn}{pandas} \PY{k}{as} \PY{n+nn}{pd}
         \PY{c+c1}{\PYZsh{} Load the iris data set from a URL.}
         \PY{n}{df} \PY{o}{=} \PY{n}{pd}\PY{o}{.}\PY{n}{read\PYZus{}csv} \PY{p}{(}\PY{l+s+s2}{\PYZdq{}}\PY{l+s+s2}{https://raw.githubusercontent.com/uiuc\PYZhy{}cse/data\PYZhy{}fa14/gh\PYZhy{}pages/data/iris.csv}\PY{l+s+s2}{\PYZdq{}}\PY{p}{)}
         \PY{c+c1}{\PYZsh{} df means \PYZdq{}data frame\PYZdq{} \PYZhy{} a data structure in Pandas, its a 2 dimensional array. In this case}
         \PY{c+c1}{\PYZsh{} it has 4 columns of floating\PYZhy{}point values, 1 column of \PYZsq{}strings\PYZsq{}(ie. non numeric classes) and 1}
         \PY{c+c1}{\PYZsh{} \PYZdq{}index\PYZdq{} column to reference each row.}
\end{Verbatim}


    \subsubsection{this imports a basic flat file (csv) with headers, then
formats it nicely, as
follows:}\label{this-imports-a-basic-flat-file-csv-with-headers-then-formats-it-nicely-as-follows}

    \begin{Verbatim}[commandchars=\\\{\}]
{\color{incolor}In [{\color{incolor}19}]:} \PY{n}{df}
\end{Verbatim}


\begin{Verbatim}[commandchars=\\\{\}]
{\color{outcolor}Out[{\color{outcolor}19}]:}      sepal\_length  sepal\_width  petal\_length  petal\_width    species
         0             5.1          3.5           1.4          0.2     setosa
         1             4.9          3.0           1.4          0.2     setosa
         2             4.7          3.2           1.3          0.2     setosa
         3             4.6          3.1           1.5          0.2     setosa
         4             5.0          3.6           1.4          0.2     setosa
         5             5.4          3.9           1.7          0.4     setosa
         6             4.6          3.4           1.4          0.3     setosa
         7             5.0          3.4           1.5          0.2     setosa
         8             4.4          2.9           1.4          0.2     setosa
         9             4.9          3.1           1.5          0.1     setosa
         10            5.4          3.7           1.5          0.2     setosa
         11            4.8          3.4           1.6          0.2     setosa
         12            4.8          3.0           1.4          0.1     setosa
         13            4.3          3.0           1.1          0.1     setosa
         14            5.8          4.0           1.2          0.2     setosa
         15            5.7          4.4           1.5          0.4     setosa
         16            5.4          3.9           1.3          0.4     setosa
         17            5.1          3.5           1.4          0.3     setosa
         18            5.7          3.8           1.7          0.3     setosa
         19            5.1          3.8           1.5          0.3     setosa
         20            5.4          3.4           1.7          0.2     setosa
         21            5.1          3.7           1.5          0.4     setosa
         22            4.6          3.6           1.0          0.2     setosa
         23            5.1          3.3           1.7          0.5     setosa
         24            4.8          3.4           1.9          0.2     setosa
         25            5.0          3.0           1.6          0.2     setosa
         26            5.0          3.4           1.6          0.4     setosa
         27            5.2          3.5           1.5          0.2     setosa
         28            5.2          3.4           1.4          0.2     setosa
         29            4.7          3.2           1.6          0.2     setosa
         ..            {\ldots}          {\ldots}           {\ldots}          {\ldots}        {\ldots}
         120           6.9          3.2           5.7          2.3  virginica
         121           5.6          2.8           4.9          2.0  virginica
         122           7.7          2.8           6.7          2.0  virginica
         123           6.3          2.7           4.9          1.8  virginica
         124           6.7          3.3           5.7          2.1  virginica
         125           7.2          3.2           6.0          1.8  virginica
         126           6.2          2.8           4.8          1.8  virginica
         127           6.1          3.0           4.9          1.8  virginica
         128           6.4          2.8           5.6          2.1  virginica
         129           7.2          3.0           5.8          1.6  virginica
         130           7.4          2.8           6.1          1.9  virginica
         131           7.9          3.8           6.4          2.0  virginica
         132           6.4          2.8           5.6          2.2  virginica
         133           6.3          2.8           5.1          1.5  virginica
         134           6.1          2.6           5.6          1.4  virginica
         135           7.7          3.0           6.1          2.3  virginica
         136           6.3          3.4           5.6          2.4  virginica
         137           6.4          3.1           5.5          1.8  virginica
         138           6.0          3.0           4.8          1.8  virginica
         139           6.9          3.1           5.4          2.1  virginica
         140           6.7          3.1           5.6          2.4  virginica
         141           6.9          3.1           5.1          2.3  virginica
         142           5.8          2.7           5.1          1.9  virginica
         143           6.8          3.2           5.9          2.3  virginica
         144           6.7          3.3           5.7          2.5  virginica
         145           6.7          3.0           5.2          2.3  virginica
         146           6.3          2.5           5.0          1.9  virginica
         147           6.5          3.0           5.2          2.0  virginica
         148           6.2          3.4           5.4          2.3  virginica
         149           5.9          3.0           5.1          1.8  virginica
         
         [150 rows x 5 columns]
\end{Verbatim}
            
    \paragraph{Pandas 'looks at' the values in the rows \& column first, to
figure out how best to format the table - eg. it detects floating-point
values, strings etc. It also looks at the first row to see if it appears
to be headers, rather than
values.}\label{pandas-looks-at-the-values-in-the-rows-column-first-to-figure-out-how-best-to-format-the-table---eg.-it-detects-floating-point-values-strings-etc.-it-also-looks-at-the-first-row-to-see-if-it-appears-to-be-headers-rather-than-values.}

In the above table, it added row labels (1-149) and although they look
like an ordered index, they are simply labels.

In this respect, Pandas is more sophisticated than numpy - which can
only handle a dataframe of homogeneous types - eg. every cell is a
floating-point value, or a string, etc. It cannot detect different data
types.

    \subsection{To make sub-selections from a dataframe in
Pandas}\label{to-make-sub-selections-from-a-dataframe-in-pandas}

\paragraph{columns and/or rows may have labels, or not. You can slice \&
name them with
Pandas}\label{columns-andor-rows-may-have-labels-or-not.-you-can-slice-name-them-with-pandas}

    \subsection{Selecting Rows and
Columns}\label{selecting-rows-and-columns}

    \begin{Verbatim}[commandchars=\\\{\}]
{\color{incolor}In [{\color{incolor}20}]:} \PY{n}{df}\PY{p}{[}\PY{p}{[}\PY{l+s+s1}{\PYZsq{}}\PY{l+s+s1}{petal\PYZus{}length}\PY{l+s+s1}{\PYZsq{}}\PY{p}{,} \PY{l+s+s1}{\PYZsq{}}\PY{l+s+s1}{species}\PY{l+s+s1}{\PYZsq{}}\PY{p}{]}\PY{p}{]}
\end{Verbatim}


\begin{Verbatim}[commandchars=\\\{\}]
{\color{outcolor}Out[{\color{outcolor}20}]:}      petal\_length    species
         0             1.4     setosa
         1             1.4     setosa
         2             1.3     setosa
         3             1.5     setosa
         4             1.4     setosa
         5             1.7     setosa
         6             1.4     setosa
         7             1.5     setosa
         8             1.4     setosa
         9             1.5     setosa
         10            1.5     setosa
         11            1.6     setosa
         12            1.4     setosa
         13            1.1     setosa
         14            1.2     setosa
         15            1.5     setosa
         16            1.3     setosa
         17            1.4     setosa
         18            1.7     setosa
         19            1.5     setosa
         20            1.7     setosa
         21            1.5     setosa
         22            1.0     setosa
         23            1.7     setosa
         24            1.9     setosa
         25            1.6     setosa
         26            1.6     setosa
         27            1.5     setosa
         28            1.4     setosa
         29            1.6     setosa
         ..            {\ldots}        {\ldots}
         120           5.7  virginica
         121           4.9  virginica
         122           6.7  virginica
         123           4.9  virginica
         124           5.7  virginica
         125           6.0  virginica
         126           4.8  virginica
         127           4.9  virginica
         128           5.6  virginica
         129           5.8  virginica
         130           6.1  virginica
         131           6.4  virginica
         132           5.6  virginica
         133           5.1  virginica
         134           5.6  virginica
         135           6.1  virginica
         136           5.6  virginica
         137           5.5  virginica
         138           4.8  virginica
         139           5.4  virginica
         140           5.6  virginica
         141           5.1  virginica
         142           5.1  virginica
         143           5.9  virginica
         144           5.7  virginica
         145           5.2  virginica
         146           5.0  virginica
         147           5.2  virginica
         148           5.4  virginica
         149           5.1  virginica
         
         [150 rows x 2 columns]
\end{Verbatim}
            
    \begin{Verbatim}[commandchars=\\\{\}]
{\color{incolor}In [{\color{incolor}21}]:} \PY{n}{df}\PY{p}{[}\PY{l+m+mi}{2}\PY{p}{:}\PY{l+m+mi}{6}\PY{p}{]}
\end{Verbatim}


\begin{Verbatim}[commandchars=\\\{\}]
{\color{outcolor}Out[{\color{outcolor}21}]:}    sepal\_length  sepal\_width  petal\_length  petal\_width species
         2           4.7          3.2           1.3          0.2  setosa
         3           4.6          3.1           1.5          0.2  setosa
         4           5.0          3.6           1.4          0.2  setosa
         5           5.4          3.9           1.7          0.4  setosa
\end{Verbatim}
            
    just told it "give me rows 2 up to, but not including, 6".

You can also combine both types of command in one call:

    \begin{Verbatim}[commandchars=\\\{\}]
{\color{incolor}In [{\color{incolor}22}]:} \PY{n}{df}\PY{p}{[}\PY{p}{[}\PY{l+s+s1}{\PYZsq{}}\PY{l+s+s1}{petal\PYZus{}length}\PY{l+s+s1}{\PYZsq{}}\PY{p}{,} \PY{l+s+s1}{\PYZsq{}}\PY{l+s+s1}{species}\PY{l+s+s1}{\PYZsq{}}\PY{p}{]}\PY{p}{]}\PY{p}{[}\PY{l+m+mi}{2}\PY{p}{:}\PY{l+m+mi}{6}\PY{p}{]}
\end{Verbatim}


\begin{Verbatim}[commandchars=\\\{\}]
{\color{outcolor}Out[{\color{outcolor}22}]:}    petal\_length species
         2           1.3  setosa
         3           1.5  setosa
         4           1.4  setosa
         5           1.7  setosa
\end{Verbatim}
            
    \section{However !}\label{however}

\subsubsection{this notation approach is not recommended - can cause
issues. The double {[}{[} {]}{]} is a clue that you're taking chances
....}\label{this-notation-approach-is-not-recommended---can-cause-issues.-the-double-is-a-clue-that-youre-taking-chances-....}

    \subsubsection{As safer approach to use loc and
iloc}\label{as-safer-approach-to-use-loc-and-iloc}

    \subsubsection{loc uses labels, iloc uses position
....}\label{loc-uses-labels-iloc-uses-position-....}

    \begin{Verbatim}[commandchars=\\\{\}]
{\color{incolor}In [{\color{incolor}23}]:} \PY{n}{df}\PY{o}{.}\PY{n}{loc}\PY{p}{[}\PY{l+m+mi}{2}\PY{p}{:}\PY{l+m+mi}{6}\PY{p}{]}
\end{Verbatim}


\begin{Verbatim}[commandchars=\\\{\}]
{\color{outcolor}Out[{\color{outcolor}23}]:}    sepal\_length  sepal\_width  petal\_length  petal\_width species
         2           4.7          3.2           1.3          0.2  setosa
         3           4.6          3.1           1.5          0.2  setosa
         4           5.0          3.6           1.4          0.2  setosa
         5           5.4          3.9           1.7          0.4  setosa
         6           4.6          3.4           1.4          0.3  setosa
\end{Verbatim}
            
    \subsubsection{it returned rows 2 to 6 because it looks at the labels -
and those row numbers are labels, rather than values. If we had used
iloc there, it would have excluded row
6.}\label{it-returned-rows-2-to-6-because-it-looks-at-the-labels---and-those-row-numbers-are-labels-rather-than-values.-if-we-had-used-iloc-there-it-would-have-excluded-row-6.}

\subsubsection{Now - to get a column
back:}\label{now---to-get-a-column-back}

    \begin{Verbatim}[commandchars=\\\{\}]
{\color{incolor}In [{\color{incolor}24}]:} \PY{n}{df}\PY{o}{.}\PY{n}{loc}\PY{p}{[}\PY{p}{:}\PY{p}{,} \PY{l+s+s1}{\PYZsq{}}\PY{l+s+s1}{species}\PY{l+s+s1}{\PYZsq{}}\PY{p}{]}
\end{Verbatim}


\begin{Verbatim}[commandchars=\\\{\}]
{\color{outcolor}Out[{\color{outcolor}24}]:} 0         setosa
         1         setosa
         2         setosa
         3         setosa
         4         setosa
         5         setosa
         6         setosa
         7         setosa
         8         setosa
         9         setosa
         10        setosa
         11        setosa
         12        setosa
         13        setosa
         14        setosa
         15        setosa
         16        setosa
         17        setosa
         18        setosa
         19        setosa
         20        setosa
         21        setosa
         22        setosa
         23        setosa
         24        setosa
         25        setosa
         26        setosa
         27        setosa
         28        setosa
         29        setosa
                  {\ldots}    
         120    virginica
         121    virginica
         122    virginica
         123    virginica
         124    virginica
         125    virginica
         126    virginica
         127    virginica
         128    virginica
         129    virginica
         130    virginica
         131    virginica
         132    virginica
         133    virginica
         134    virginica
         135    virginica
         136    virginica
         137    virginica
         138    virginica
         139    virginica
         140    virginica
         141    virginica
         142    virginica
         143    virginica
         144    virginica
         145    virginica
         146    virginica
         147    virginica
         148    virginica
         149    virginica
         Name: species, Length: 150, dtype: object
\end{Verbatim}
            
    \subsubsection{in that command, the colon meant "all" - so, all rows,
and just the 'species' column. Now try this approach
....}\label{in-that-command-the-colon-meant-all---so-all-rows-and-just-the-species-column.-now-try-this-approach-....}

    \begin{Verbatim}[commandchars=\\\{\}]
{\color{incolor}In [{\color{incolor}25}]:} \PY{n}{df}\PY{o}{.}\PY{n}{loc}\PY{p}{[}\PY{l+m+mi}{2}\PY{p}{:}\PY{l+m+mi}{6}\PY{p}{,} \PY{p}{[}\PY{l+s+s1}{\PYZsq{}}\PY{l+s+s1}{sepal\PYZus{}length}\PY{l+s+s1}{\PYZsq{}}\PY{p}{,} \PY{l+s+s1}{\PYZsq{}}\PY{l+s+s1}{species}\PY{l+s+s1}{\PYZsq{}}\PY{p}{]}\PY{p}{]}
\end{Verbatim}


\begin{Verbatim}[commandchars=\\\{\}]
{\color{outcolor}Out[{\color{outcolor}25}]:}    sepal\_length species
         2           4.7  setosa
         3           4.6  setosa
         4           5.0  setosa
         5           5.4  setosa
         6           4.6  setosa
\end{Verbatim}
            
    \subsubsection{so "give me this list of column labels, and this list of
row
labels"}\label{so-give-me-this-list-of-column-labels-and-this-list-of-row-labels}

\subsubsection{note that the original row labels persist (eg. first one
is '2') - be aware of this effect when using loc. Whereas ... iloc
returns
position:}\label{note-that-the-original-row-labels-persist-eg.-first-one-is-2---be-aware-of-this-effect-when-using-loc.-whereas-...-iloc-returns-position}

    \begin{Verbatim}[commandchars=\\\{\}]
{\color{incolor}In [{\color{incolor}26}]:} \PY{n}{df}\PY{o}{.}\PY{n}{iloc}\PY{p}{[}\PY{l+m+mi}{2}\PY{p}{]}
\end{Verbatim}


\begin{Verbatim}[commandchars=\\\{\}]
{\color{outcolor}Out[{\color{outcolor}26}]:} sepal\_length       4.7
         sepal\_width        3.2
         petal\_length       1.3
         petal\_width        0.2
         species         setosa
         Name: 2, dtype: object
\end{Verbatim}
            
    \paragraph{that returned the values from the row position 2 (remember, 0
is the
first)}\label{that-returned-the-values-from-the-row-position-2-remember-0-is-the-first}

    \begin{Verbatim}[commandchars=\\\{\}]
{\color{incolor}In [{\color{incolor}27}]:} \PY{n}{df}\PY{o}{.}\PY{n}{iloc}\PY{p}{[}\PY{l+m+mi}{2}\PY{p}{:}\PY{l+m+mi}{4}\PY{p}{,}\PY{l+m+mi}{1}\PY{p}{]}
\end{Verbatim}


\begin{Verbatim}[commandchars=\\\{\}]
{\color{outcolor}Out[{\color{outcolor}27}]:} 2    3.2
         3    3.1
         Name: sepal\_width, dtype: float64
\end{Verbatim}
            
    \subsubsection{that returned the values from rows in position 2 to 4,
and column position 1 (which is the 2nd
column)}\label{that-returned-the-values-from-rows-in-position-2-to-4-and-column-position-1-which-is-the-2nd-column}

Now, the 'at' command can be used to return a sibgle value from an
array:

    \begin{Verbatim}[commandchars=\\\{\}]
{\color{incolor}In [{\color{incolor}28}]:} \PY{n}{df}\PY{o}{.}\PY{n}{at}\PY{p}{[}\PY{l+m+mi}{3}\PY{p}{,} \PY{l+s+s1}{\PYZsq{}}\PY{l+s+s1}{species}\PY{l+s+s1}{\PYZsq{}}\PY{p}{]}
\end{Verbatim}


\begin{Verbatim}[commandchars=\\\{\}]
{\color{outcolor}Out[{\color{outcolor}28}]:} 'setosa'
\end{Verbatim}
            
    \subsubsection{\texorpdfstring{... \emph{so, loc uses Labels, iloc uses
Positions}
!}{... so, loc uses Labels, iloc uses Positions !}}\label{so-loc-uses-labels-iloc-uses-positions}

    \subsection{Boolean Selects:}\label{boolean-selects}

    \begin{Verbatim}[commandchars=\\\{\}]
{\color{incolor}In [{\color{incolor}29}]:} \PY{n}{df}\PY{o}{.}\PY{n}{loc}\PY{p}{[}\PY{p}{:}\PY{p}{,} \PY{l+s+s1}{\PYZsq{}}\PY{l+s+s1}{species}\PY{l+s+s1}{\PYZsq{}}\PY{p}{]} \PY{o}{==} \PY{l+s+s1}{\PYZsq{}}\PY{l+s+s1}{setosa}\PY{l+s+s1}{\PYZsq{}}
\end{Verbatim}


\begin{Verbatim}[commandchars=\\\{\}]
{\color{outcolor}Out[{\color{outcolor}29}]:} 0       True
         1       True
         2       True
         3       True
         4       True
         5       True
         6       True
         7       True
         8       True
         9       True
         10      True
         11      True
         12      True
         13      True
         14      True
         15      True
         16      True
         17      True
         18      True
         19      True
         20      True
         21      True
         22      True
         23      True
         24      True
         25      True
         26      True
         27      True
         28      True
         29      True
                {\ldots}  
         120    False
         121    False
         122    False
         123    False
         124    False
         125    False
         126    False
         127    False
         128    False
         129    False
         130    False
         131    False
         132    False
         133    False
         134    False
         135    False
         136    False
         137    False
         138    False
         139    False
         140    False
         141    False
         142    False
         143    False
         144    False
         145    False
         146    False
         147    False
         148    False
         149    False
         Name: species, Length: 150, dtype: bool
\end{Verbatim}
            
    \subsubsection{== compared the value in "species" to the string "setosa"
- if its setosa, then its
TRUE.}\label{compared-the-value-in-species-to-the-string-setosa---if-its-setosa-then-its-true.}

    Now, to return just the rows with the setosa string:

    \begin{Verbatim}[commandchars=\\\{\}]
{\color{incolor}In [{\color{incolor}30}]:} \PY{n}{df}\PY{o}{.}\PY{n}{loc}\PY{p}{[}\PY{n}{df}\PY{o}{.}\PY{n}{loc}\PY{p}{[}\PY{p}{:}\PY{p}{,} \PY{l+s+s1}{\PYZsq{}}\PY{l+s+s1}{species}\PY{l+s+s1}{\PYZsq{}}\PY{p}{]} \PY{o}{==} \PY{l+s+s1}{\PYZsq{}}\PY{l+s+s1}{setosa}\PY{l+s+s1}{\PYZsq{}}\PY{p}{]}
\end{Verbatim}


\begin{Verbatim}[commandchars=\\\{\}]
{\color{outcolor}Out[{\color{outcolor}30}]:}     sepal\_length  sepal\_width  petal\_length  petal\_width species
         0            5.1          3.5           1.4          0.2  setosa
         1            4.9          3.0           1.4          0.2  setosa
         2            4.7          3.2           1.3          0.2  setosa
         3            4.6          3.1           1.5          0.2  setosa
         4            5.0          3.6           1.4          0.2  setosa
         5            5.4          3.9           1.7          0.4  setosa
         6            4.6          3.4           1.4          0.3  setosa
         7            5.0          3.4           1.5          0.2  setosa
         8            4.4          2.9           1.4          0.2  setosa
         9            4.9          3.1           1.5          0.1  setosa
         10           5.4          3.7           1.5          0.2  setosa
         11           4.8          3.4           1.6          0.2  setosa
         12           4.8          3.0           1.4          0.1  setosa
         13           4.3          3.0           1.1          0.1  setosa
         14           5.8          4.0           1.2          0.2  setosa
         15           5.7          4.4           1.5          0.4  setosa
         16           5.4          3.9           1.3          0.4  setosa
         17           5.1          3.5           1.4          0.3  setosa
         18           5.7          3.8           1.7          0.3  setosa
         19           5.1          3.8           1.5          0.3  setosa
         20           5.4          3.4           1.7          0.2  setosa
         21           5.1          3.7           1.5          0.4  setosa
         22           4.6          3.6           1.0          0.2  setosa
         23           5.1          3.3           1.7          0.5  setosa
         24           4.8          3.4           1.9          0.2  setosa
         25           5.0          3.0           1.6          0.2  setosa
         26           5.0          3.4           1.6          0.4  setosa
         27           5.2          3.5           1.5          0.2  setosa
         28           5.2          3.4           1.4          0.2  setosa
         29           4.7          3.2           1.6          0.2  setosa
         30           4.8          3.1           1.6          0.2  setosa
         31           5.4          3.4           1.5          0.4  setosa
         32           5.2          4.1           1.5          0.1  setosa
         33           5.5          4.2           1.4          0.2  setosa
         34           4.9          3.1           1.5          0.1  setosa
         35           5.0          3.2           1.2          0.2  setosa
         36           5.5          3.5           1.3          0.2  setosa
         37           4.9          3.1           1.5          0.1  setosa
         38           4.4          3.0           1.3          0.2  setosa
         39           5.1          3.4           1.5          0.2  setosa
         40           5.0          3.5           1.3          0.3  setosa
         41           4.5          2.3           1.3          0.3  setosa
         42           4.4          3.2           1.3          0.2  setosa
         43           5.0          3.5           1.6          0.6  setosa
         44           5.1          3.8           1.9          0.4  setosa
         45           4.8          3.0           1.4          0.3  setosa
         46           5.1          3.8           1.6          0.2  setosa
         47           4.6          3.2           1.4          0.2  setosa
         48           5.3          3.7           1.5          0.2  setosa
         49           5.0          3.3           1.4          0.2  setosa
\end{Verbatim}
            
    \begin{Verbatim}[commandchars=\\\{\}]
{\color{incolor}In [{\color{incolor}31}]:} \PY{n}{df}\PY{o}{.}\PY{n}{loc}\PY{p}{[}\PY{n}{df}\PY{o}{.}\PY{n}{loc}\PY{p}{[}\PY{p}{:}\PY{p}{,} \PY{l+s+s1}{\PYZsq{}}\PY{l+s+s1}{species}\PY{l+s+s1}{\PYZsq{}}\PY{p}{]} \PY{o}{==} \PY{l+s+s1}{\PYZsq{}}\PY{l+s+s1}{versicolor}\PY{l+s+s1}{\PYZsq{}}\PY{p}{]}
\end{Verbatim}


\begin{Verbatim}[commandchars=\\\{\}]
{\color{outcolor}Out[{\color{outcolor}31}]:}     sepal\_length  sepal\_width  petal\_length  petal\_width     species
         50           7.0          3.2           4.7          1.4  versicolor
         51           6.4          3.2           4.5          1.5  versicolor
         52           6.9          3.1           4.9          1.5  versicolor
         53           5.5          2.3           4.0          1.3  versicolor
         54           6.5          2.8           4.6          1.5  versicolor
         55           5.7          2.8           4.5          1.3  versicolor
         56           6.3          3.3           4.7          1.6  versicolor
         57           4.9          2.4           3.3          1.0  versicolor
         58           6.6          2.9           4.6          1.3  versicolor
         59           5.2          2.7           3.9          1.4  versicolor
         60           5.0          2.0           3.5          1.0  versicolor
         61           5.9          3.0           4.2          1.5  versicolor
         62           6.0          2.2           4.0          1.0  versicolor
         63           6.1          2.9           4.7          1.4  versicolor
         64           5.6          2.9           3.6          1.3  versicolor
         65           6.7          3.1           4.4          1.4  versicolor
         66           5.6          3.0           4.5          1.5  versicolor
         67           5.8          2.7           4.1          1.0  versicolor
         68           6.2          2.2           4.5          1.5  versicolor
         69           5.6          2.5           3.9          1.1  versicolor
         70           5.9          3.2           4.8          1.8  versicolor
         71           6.1          2.8           4.0          1.3  versicolor
         72           6.3          2.5           4.9          1.5  versicolor
         73           6.1          2.8           4.7          1.2  versicolor
         74           6.4          2.9           4.3          1.3  versicolor
         75           6.6          3.0           4.4          1.4  versicolor
         76           6.8          2.8           4.8          1.4  versicolor
         77           6.7          3.0           5.0          1.7  versicolor
         78           6.0          2.9           4.5          1.5  versicolor
         79           5.7          2.6           3.5          1.0  versicolor
         80           5.5          2.4           3.8          1.1  versicolor
         81           5.5          2.4           3.7          1.0  versicolor
         82           5.8          2.7           3.9          1.2  versicolor
         83           6.0          2.7           5.1          1.6  versicolor
         84           5.4          3.0           4.5          1.5  versicolor
         85           6.0          3.4           4.5          1.6  versicolor
         86           6.7          3.1           4.7          1.5  versicolor
         87           6.3          2.3           4.4          1.3  versicolor
         88           5.6          3.0           4.1          1.3  versicolor
         89           5.5          2.5           4.0          1.3  versicolor
         90           5.5          2.6           4.4          1.2  versicolor
         91           6.1          3.0           4.6          1.4  versicolor
         92           5.8          2.6           4.0          1.2  versicolor
         93           5.0          2.3           3.3          1.0  versicolor
         94           5.6          2.7           4.2          1.3  versicolor
         95           5.7          3.0           4.2          1.2  versicolor
         96           5.7          2.9           4.2          1.3  versicolor
         97           6.2          2.9           4.3          1.3  versicolor
         98           5.1          2.5           3.0          1.1  versicolor
         99           5.7          2.8           4.1          1.3  versicolor
\end{Verbatim}
            
    \begin{Verbatim}[commandchars=\\\{\}]
{\color{incolor}In [{\color{incolor} }]:} \PY{n}{df}\PY{o}{.}\PY{n}{loc}\PY{p}{[}\PY{n}{df}\PY{o}{.}\PY{n}{loc}\PY{p}{[}\PY{p}{:}\PY{p}{,} \PY{l+s+s1}{\PYZsq{}}\PY{l+s+s1}{species}\PY{l+s+s1}{\PYZsq{}}\PY{p}{]} \PY{o}{==} \PY{l+s+s1}{\PYZsq{}}\PY{l+s+s1}{setosa}\PY{l+s+s1}{\PYZsq{}}\PY{p}{]}
\end{Verbatim}


    \begin{Verbatim}[commandchars=\\\{\}]
{\color{incolor}In [{\color{incolor}44}]:} \PY{n}{x} \PY{o}{=} \PY{n}{df}\PY{o}{.}\PY{n}{loc}\PY{p}{[}\PY{n}{df}\PY{o}{.}\PY{n}{loc}\PY{p}{[}\PY{p}{:}\PY{p}{,} \PY{l+s+s1}{\PYZsq{}}\PY{l+s+s1}{species}\PY{l+s+s1}{\PYZsq{}}\PY{p}{]} \PY{o}{==} \PY{l+s+s1}{\PYZsq{}}\PY{l+s+s1}{virginica}\PY{l+s+s1}{\PYZsq{}}\PY{p}{]}
         \PY{n}{x}
\end{Verbatim}


\begin{Verbatim}[commandchars=\\\{\}]
{\color{outcolor}Out[{\color{outcolor}44}]:}      sepal\_length  sepal\_width  petal\_length  petal\_width    species
         100           6.3          3.3           6.0          2.5  virginica
         101           5.8          2.7           5.1          1.9  virginica
         102           7.1          3.0           5.9          2.1  virginica
         103           6.3          2.9           5.6          1.8  virginica
         104           6.5          3.0           5.8          2.2  virginica
         105           7.6          3.0           6.6          2.1  virginica
         106           4.9          2.5           4.5          1.7  virginica
         107           7.3          2.9           6.3          1.8  virginica
         108           6.7          2.5           5.8          1.8  virginica
         109           7.2          3.6           6.1          2.5  virginica
         110           6.5          3.2           5.1          2.0  virginica
         111           6.4          2.7           5.3          1.9  virginica
         112           6.8          3.0           5.5          2.1  virginica
         113           5.7          2.5           5.0          2.0  virginica
         114           5.8          2.8           5.1          2.4  virginica
         115           6.4          3.2           5.3          2.3  virginica
         116           6.5          3.0           5.5          1.8  virginica
         117           7.7          3.8           6.7          2.2  virginica
         118           7.7          2.6           6.9          2.3  virginica
         119           6.0          2.2           5.0          1.5  virginica
         120           6.9          3.2           5.7          2.3  virginica
         121           5.6          2.8           4.9          2.0  virginica
         122           7.7          2.8           6.7          2.0  virginica
         123           6.3          2.7           4.9          1.8  virginica
         124           6.7          3.3           5.7          2.1  virginica
         125           7.2          3.2           6.0          1.8  virginica
         126           6.2          2.8           4.8          1.8  virginica
         127           6.1          3.0           4.9          1.8  virginica
         128           6.4          2.8           5.6          2.1  virginica
         129           7.2          3.0           5.8          1.6  virginica
         130           7.4          2.8           6.1          1.9  virginica
         131           7.9          3.8           6.4          2.0  virginica
         132           6.4          2.8           5.6          2.2  virginica
         133           6.3          2.8           5.1          1.5  virginica
         134           6.1          2.6           5.6          1.4  virginica
         135           7.7          3.0           6.1          2.3  virginica
         136           6.3          3.4           5.6          2.4  virginica
         137           6.4          3.1           5.5          1.8  virginica
         138           6.0          3.0           4.8          1.8  virginica
         139           6.9          3.1           5.4          2.1  virginica
         140           6.7          3.1           5.6          2.4  virginica
         141           6.9          3.1           5.1          2.3  virginica
         142           5.8          2.7           5.1          1.9  virginica
         143           6.8          3.2           5.9          2.3  virginica
         144           6.7          3.3           5.7          2.5  virginica
         145           6.7          3.0           5.2          2.3  virginica
         146           6.3          2.5           5.0          1.9  virginica
         147           6.5          3.0           5.2          2.0  virginica
         148           6.2          3.4           5.4          2.3  virginica
         149           5.9          3.0           5.1          1.8  virginica
\end{Verbatim}
            
    \begin{Verbatim}[commandchars=\\\{\}]
{\color{incolor}In [{\color{incolor}33}]:} \PY{n}{x}\PY{o}{.}\PY{n}{loc}\PY{p}{[}\PY{l+m+mi}{51}\PY{p}{]}
\end{Verbatim}


\begin{Verbatim}[commandchars=\\\{\}]
{\color{outcolor}Out[{\color{outcolor}33}]:} sepal\_length           6.4
         sepal\_width            3.2
         petal\_length           4.5
         petal\_width            1.5
         species         versicolor
         Name: 51, dtype: object
\end{Verbatim}
            
    that returned the values in row \emph{labeled} 51. Using iloc to get the
same result requires you to specify the \emph{position} of that row in
the new dataframe, which is actually pos 1.

    \begin{Verbatim}[commandchars=\\\{\}]
{\color{incolor}In [{\color{incolor}34}]:} \PY{n}{x}\PY{o}{.}\PY{n}{iloc}\PY{p}{[}\PY{l+m+mi}{1}\PY{p}{]}
\end{Verbatim}


\begin{Verbatim}[commandchars=\\\{\}]
{\color{outcolor}Out[{\color{outcolor}34}]:} sepal\_length           6.4
         sepal\_width            3.2
         petal\_length           4.5
         petal\_width            1.5
         species         versicolor
         Name: 51, dtype: object
\end{Verbatim}
            
    \subsection{Summary Statistics}\label{summary-statistics}

    \paragraph{to take a quick look at a dataframe (maybe an imported CSV
file) to see if it looks clean \& structured, its useful to use these
'head', 'tail' and other
commands.}\label{to-take-a-quick-look-at-a-dataframe-maybe-an-imported-csv-file-to-see-if-it-looks-clean-structured-its-useful-to-use-these-head-tail-and-other-commands.}

    \begin{Verbatim}[commandchars=\\\{\}]
{\color{incolor}In [{\color{incolor}35}]:} \PY{n}{df}\PY{o}{.}\PY{n}{head}\PY{p}{(}\PY{p}{)}
\end{Verbatim}


\begin{Verbatim}[commandchars=\\\{\}]
{\color{outcolor}Out[{\color{outcolor}35}]:}    sepal\_length  sepal\_width  petal\_length  petal\_width species
         0           5.1          3.5           1.4          0.2  setosa
         1           4.9          3.0           1.4          0.2  setosa
         2           4.7          3.2           1.3          0.2  setosa
         3           4.6          3.1           1.5          0.2  setosa
         4           5.0          3.6           1.4          0.2  setosa
\end{Verbatim}
            
    \begin{Verbatim}[commandchars=\\\{\}]
{\color{incolor}In [{\color{incolor}36}]:} \PY{n}{df}\PY{o}{.}\PY{n}{tail}\PY{p}{(}\PY{p}{)}
\end{Verbatim}


\begin{Verbatim}[commandchars=\\\{\}]
{\color{outcolor}Out[{\color{outcolor}36}]:}      sepal\_length  sepal\_width  petal\_length  petal\_width    species
         145           6.7          3.0           5.2          2.3  virginica
         146           6.3          2.5           5.0          1.9  virginica
         147           6.5          3.0           5.2          2.0  virginica
         148           6.2          3.4           5.4          2.3  virginica
         149           5.9          3.0           5.1          1.8  virginica
\end{Verbatim}
            
    \begin{Verbatim}[commandchars=\\\{\}]
{\color{incolor}In [{\color{incolor}37}]:} \PY{n}{df}\PY{o}{.}\PY{n}{describe}\PY{p}{(}\PY{p}{)}
\end{Verbatim}


\begin{Verbatim}[commandchars=\\\{\}]
{\color{outcolor}Out[{\color{outcolor}37}]:}        sepal\_length  sepal\_width  petal\_length  petal\_width
         count    150.000000   150.000000    150.000000   150.000000
         mean       5.843333     3.054000      3.758667     1.198667
         std        0.828066     0.433594      1.764420     0.763161
         min        4.300000     2.000000      1.000000     0.100000
         25\%        5.100000     2.800000      1.600000     0.300000
         50\%        5.800000     3.000000      4.350000     1.300000
         75\%        6.400000     3.300000      5.100000     1.800000
         max        7.900000     4.400000      6.900000     2.500000
\end{Verbatim}
            
    \begin{Verbatim}[commandchars=\\\{\}]
{\color{incolor}In [{\color{incolor}38}]:} \PY{n}{df}\PY{o}{.}\PY{n}{mean}\PY{p}{(}\PY{p}{)}
\end{Verbatim}


\begin{Verbatim}[commandchars=\\\{\}]
{\color{outcolor}Out[{\color{outcolor}38}]:} sepal\_length    5.843333
         sepal\_width     3.054000
         petal\_length    3.758667
         petal\_width     1.198667
         dtype: float64
\end{Verbatim}
            
    \subsubsection{btw - the 50\% above is the 50th percentile = the median.
So 50\% of the values are less than this
datapoint.}\label{btw---the-50-above-is-the-50th-percentile-the-median.-so-50-of-the-values-are-less-than-this-datapoint.}

    you may want to subtract the mean from all the values in order to
'centre' (or "whiting") the dataset - then each value is delta from the
mean

    \subsubsection{PLOTS}\label{plots}

    \begin{Verbatim}[commandchars=\\\{\}]
{\color{incolor}In [{\color{incolor}39}]:} \PY{k+kn}{import} \PY{n+nn}{seaborn} \PY{k}{as} \PY{n+nn}{sns}
\end{Verbatim}


    \begin{Verbatim}[commandchars=\\\{\}]
{\color{incolor}In [{\color{incolor}40}]:} \PY{n}{sns}\PY{o}{.}\PY{n}{pairplot}\PY{p}{(}\PY{n}{df}\PY{p}{)}
\end{Verbatim}


\begin{Verbatim}[commandchars=\\\{\}]
{\color{outcolor}Out[{\color{outcolor}40}]:} <seaborn.axisgrid.PairGrid at 0x11bd0ab00>
\end{Verbatim}
            
    \begin{center}
    \adjustimage{max size={0.9\linewidth}{0.9\paperheight}}{output_101_1.png}
    \end{center}
    { \hspace*{\fill} \\}
    
    Then, by adding the 'hue' command, you can distinguish the varieties of
iris in the dataset by means of colour:

    \begin{Verbatim}[commandchars=\\\{\}]
{\color{incolor}In [{\color{incolor}41}]:} \PY{n}{sns}\PY{o}{.}\PY{n}{pairplot}\PY{p}{(}\PY{n}{df}\PY{p}{,} \PY{n}{hue}\PY{o}{=}\PY{l+s+s1}{\PYZsq{}}\PY{l+s+s1}{species}\PY{l+s+s1}{\PYZsq{}}\PY{p}{)}
\end{Verbatim}


\begin{Verbatim}[commandchars=\\\{\}]
{\color{outcolor}Out[{\color{outcolor}41}]:} <seaborn.axisgrid.PairGrid at 0x1a1e8eab38>
\end{Verbatim}
            
    \begin{center}
    \adjustimage{max size={0.9\linewidth}{0.9\paperheight}}{output_103_1.png}
    \end{center}
    { \hspace*{\fill} \\}
    
    .... that command returned histograms in the diagonal position.

    \begin{Verbatim}[commandchars=\\\{\}]
{\color{incolor}In [{\color{incolor}42}]:} \PY{n}{sns}\PY{o}{.}\PY{n}{pairplot}\PY{p}{(}\PY{n}{df}\PY{p}{,} \PY{n}{hue}\PY{o}{=}\PY{l+s+s1}{\PYZsq{}}\PY{l+s+s1}{species}\PY{l+s+s1}{\PYZsq{}}\PY{p}{,} \PY{n}{diag\PYZus{}kind}\PY{o}{=} \PY{l+s+s1}{\PYZsq{}}\PY{l+s+s1}{kde}\PY{l+s+s1}{\PYZsq{}}\PY{p}{)}
\end{Verbatim}


\begin{Verbatim}[commandchars=\\\{\}]
{\color{outcolor}Out[{\color{outcolor}42}]:} <seaborn.axisgrid.PairGrid at 0x1a1f9e76d8>
\end{Verbatim}
            
    \begin{center}
    \adjustimage{max size={0.9\linewidth}{0.9\paperheight}}{output_105_1.png}
    \end{center}
    { \hspace*{\fill} \\}
    
    ... that command returned curves in the diagonal position.


    % Add a bibliography block to the postdoc
    
    
    
    \end{document}
