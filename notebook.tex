
% Default to the notebook output style

    


% Inherit from the specified cell style.




    
\documentclass[11pt]{article}

    
    
    \usepackage[T1]{fontenc}
    % Nicer default font (+ math font) than Computer Modern for most use cases
    \usepackage{mathpazo}

    % Basic figure setup, for now with no caption control since it's done
    % automatically by Pandoc (which extracts ![](path) syntax from Markdown).
    \usepackage{graphicx}
    % We will generate all images so they have a width \maxwidth. This means
    % that they will get their normal width if they fit onto the page, but
    % are scaled down if they would overflow the margins.
    \makeatletter
    \def\maxwidth{\ifdim\Gin@nat@width>\linewidth\linewidth
    \else\Gin@nat@width\fi}
    \makeatother
    \let\Oldincludegraphics\includegraphics
    % Set max figure width to be 80% of text width, for now hardcoded.
    \renewcommand{\includegraphics}[1]{\Oldincludegraphics[width=.8\maxwidth]{#1}}
    % Ensure that by default, figures have no caption (until we provide a
    % proper Figure object with a Caption API and a way to capture that
    % in the conversion process - todo).
    \usepackage{caption}
    \DeclareCaptionLabelFormat{nolabel}{}
    \captionsetup{labelformat=nolabel}

    \usepackage{adjustbox} % Used to constrain images to a maximum size 
    \usepackage{xcolor} % Allow colors to be defined
    \usepackage{enumerate} % Needed for markdown enumerations to work
    \usepackage{geometry} % Used to adjust the document margins
    \usepackage{amsmath} % Equations
    \usepackage{amssymb} % Equations
    \usepackage{textcomp} % defines textquotesingle
    % Hack from http://tex.stackexchange.com/a/47451/13684:
    \AtBeginDocument{%
        \def\PYZsq{\textquotesingle}% Upright quotes in Pygmentized code
    }
    \usepackage{upquote} % Upright quotes for verbatim code
    \usepackage{eurosym} % defines \euro
    \usepackage[mathletters]{ucs} % Extended unicode (utf-8) support
    \usepackage[utf8x]{inputenc} % Allow utf-8 characters in the tex document
    \usepackage{fancyvrb} % verbatim replacement that allows latex
    \usepackage{grffile} % extends the file name processing of package graphics 
                         % to support a larger range 
    % The hyperref package gives us a pdf with properly built
    % internal navigation ('pdf bookmarks' for the table of contents,
    % internal cross-reference links, web links for URLs, etc.)
    \usepackage{hyperref}
    \usepackage{longtable} % longtable support required by pandoc >1.10
    \usepackage{booktabs}  % table support for pandoc > 1.12.2
    \usepackage[inline]{enumitem} % IRkernel/repr support (it uses the enumerate* environment)
    \usepackage[normalem]{ulem} % ulem is needed to support strikethroughs (\sout)
                                % normalem makes italics be italics, not underlines
    

    
    
    % Colors for the hyperref package
    \definecolor{urlcolor}{rgb}{0,.145,.698}
    \definecolor{linkcolor}{rgb}{.71,0.21,0.01}
    \definecolor{citecolor}{rgb}{.12,.54,.11}

    % ANSI colors
    \definecolor{ansi-black}{HTML}{3E424D}
    \definecolor{ansi-black-intense}{HTML}{282C36}
    \definecolor{ansi-red}{HTML}{E75C58}
    \definecolor{ansi-red-intense}{HTML}{B22B31}
    \definecolor{ansi-green}{HTML}{00A250}
    \definecolor{ansi-green-intense}{HTML}{007427}
    \definecolor{ansi-yellow}{HTML}{DDB62B}
    \definecolor{ansi-yellow-intense}{HTML}{B27D12}
    \definecolor{ansi-blue}{HTML}{208FFB}
    \definecolor{ansi-blue-intense}{HTML}{0065CA}
    \definecolor{ansi-magenta}{HTML}{D160C4}
    \definecolor{ansi-magenta-intense}{HTML}{A03196}
    \definecolor{ansi-cyan}{HTML}{60C6C8}
    \definecolor{ansi-cyan-intense}{HTML}{258F8F}
    \definecolor{ansi-white}{HTML}{C5C1B4}
    \definecolor{ansi-white-intense}{HTML}{A1A6B2}

    % commands and environments needed by pandoc snippets
    % extracted from the output of `pandoc -s`
    \providecommand{\tightlist}{%
      \setlength{\itemsep}{0pt}\setlength{\parskip}{0pt}}
    \DefineVerbatimEnvironment{Highlighting}{Verbatim}{commandchars=\\\{\}}
    % Add ',fontsize=\small' for more characters per line
    \newenvironment{Shaded}{}{}
    \newcommand{\KeywordTok}[1]{\textcolor[rgb]{0.00,0.44,0.13}{\textbf{{#1}}}}
    \newcommand{\DataTypeTok}[1]{\textcolor[rgb]{0.56,0.13,0.00}{{#1}}}
    \newcommand{\DecValTok}[1]{\textcolor[rgb]{0.25,0.63,0.44}{{#1}}}
    \newcommand{\BaseNTok}[1]{\textcolor[rgb]{0.25,0.63,0.44}{{#1}}}
    \newcommand{\FloatTok}[1]{\textcolor[rgb]{0.25,0.63,0.44}{{#1}}}
    \newcommand{\CharTok}[1]{\textcolor[rgb]{0.25,0.44,0.63}{{#1}}}
    \newcommand{\StringTok}[1]{\textcolor[rgb]{0.25,0.44,0.63}{{#1}}}
    \newcommand{\CommentTok}[1]{\textcolor[rgb]{0.38,0.63,0.69}{\textit{{#1}}}}
    \newcommand{\OtherTok}[1]{\textcolor[rgb]{0.00,0.44,0.13}{{#1}}}
    \newcommand{\AlertTok}[1]{\textcolor[rgb]{1.00,0.00,0.00}{\textbf{{#1}}}}
    \newcommand{\FunctionTok}[1]{\textcolor[rgb]{0.02,0.16,0.49}{{#1}}}
    \newcommand{\RegionMarkerTok}[1]{{#1}}
    \newcommand{\ErrorTok}[1]{\textcolor[rgb]{1.00,0.00,0.00}{\textbf{{#1}}}}
    \newcommand{\NormalTok}[1]{{#1}}
    
    % Additional commands for more recent versions of Pandoc
    \newcommand{\ConstantTok}[1]{\textcolor[rgb]{0.53,0.00,0.00}{{#1}}}
    \newcommand{\SpecialCharTok}[1]{\textcolor[rgb]{0.25,0.44,0.63}{{#1}}}
    \newcommand{\VerbatimStringTok}[1]{\textcolor[rgb]{0.25,0.44,0.63}{{#1}}}
    \newcommand{\SpecialStringTok}[1]{\textcolor[rgb]{0.73,0.40,0.53}{{#1}}}
    \newcommand{\ImportTok}[1]{{#1}}
    \newcommand{\DocumentationTok}[1]{\textcolor[rgb]{0.73,0.13,0.13}{\textit{{#1}}}}
    \newcommand{\AnnotationTok}[1]{\textcolor[rgb]{0.38,0.63,0.69}{\textbf{\textit{{#1}}}}}
    \newcommand{\CommentVarTok}[1]{\textcolor[rgb]{0.38,0.63,0.69}{\textbf{\textit{{#1}}}}}
    \newcommand{\VariableTok}[1]{\textcolor[rgb]{0.10,0.09,0.49}{{#1}}}
    \newcommand{\ControlFlowTok}[1]{\textcolor[rgb]{0.00,0.44,0.13}{\textbf{{#1}}}}
    \newcommand{\OperatorTok}[1]{\textcolor[rgb]{0.40,0.40,0.40}{{#1}}}
    \newcommand{\BuiltInTok}[1]{{#1}}
    \newcommand{\ExtensionTok}[1]{{#1}}
    \newcommand{\PreprocessorTok}[1]{\textcolor[rgb]{0.74,0.48,0.00}{{#1}}}
    \newcommand{\AttributeTok}[1]{\textcolor[rgb]{0.49,0.56,0.16}{{#1}}}
    \newcommand{\InformationTok}[1]{\textcolor[rgb]{0.38,0.63,0.69}{\textbf{\textit{{#1}}}}}
    \newcommand{\WarningTok}[1]{\textcolor[rgb]{0.38,0.63,0.69}{\textbf{\textit{{#1}}}}}
    
    
    % Define a nice break command that doesn't care if a line doesn't already
    % exist.
    \def\br{\hspace*{\fill} \\* }
    % Math Jax compatability definitions
    \def\gt{>}
    \def\lt{<}
    % Document parameters
    \title{numpy-random}
    
    
    

    % Pygments definitions
    
\makeatletter
\def\PY@reset{\let\PY@it=\relax \let\PY@bf=\relax%
    \let\PY@ul=\relax \let\PY@tc=\relax%
    \let\PY@bc=\relax \let\PY@ff=\relax}
\def\PY@tok#1{\csname PY@tok@#1\endcsname}
\def\PY@toks#1+{\ifx\relax#1\empty\else%
    \PY@tok{#1}\expandafter\PY@toks\fi}
\def\PY@do#1{\PY@bc{\PY@tc{\PY@ul{%
    \PY@it{\PY@bf{\PY@ff{#1}}}}}}}
\def\PY#1#2{\PY@reset\PY@toks#1+\relax+\PY@do{#2}}

\expandafter\def\csname PY@tok@w\endcsname{\def\PY@tc##1{\textcolor[rgb]{0.73,0.73,0.73}{##1}}}
\expandafter\def\csname PY@tok@c\endcsname{\let\PY@it=\textit\def\PY@tc##1{\textcolor[rgb]{0.25,0.50,0.50}{##1}}}
\expandafter\def\csname PY@tok@cp\endcsname{\def\PY@tc##1{\textcolor[rgb]{0.74,0.48,0.00}{##1}}}
\expandafter\def\csname PY@tok@k\endcsname{\let\PY@bf=\textbf\def\PY@tc##1{\textcolor[rgb]{0.00,0.50,0.00}{##1}}}
\expandafter\def\csname PY@tok@kp\endcsname{\def\PY@tc##1{\textcolor[rgb]{0.00,0.50,0.00}{##1}}}
\expandafter\def\csname PY@tok@kt\endcsname{\def\PY@tc##1{\textcolor[rgb]{0.69,0.00,0.25}{##1}}}
\expandafter\def\csname PY@tok@o\endcsname{\def\PY@tc##1{\textcolor[rgb]{0.40,0.40,0.40}{##1}}}
\expandafter\def\csname PY@tok@ow\endcsname{\let\PY@bf=\textbf\def\PY@tc##1{\textcolor[rgb]{0.67,0.13,1.00}{##1}}}
\expandafter\def\csname PY@tok@nb\endcsname{\def\PY@tc##1{\textcolor[rgb]{0.00,0.50,0.00}{##1}}}
\expandafter\def\csname PY@tok@nf\endcsname{\def\PY@tc##1{\textcolor[rgb]{0.00,0.00,1.00}{##1}}}
\expandafter\def\csname PY@tok@nc\endcsname{\let\PY@bf=\textbf\def\PY@tc##1{\textcolor[rgb]{0.00,0.00,1.00}{##1}}}
\expandafter\def\csname PY@tok@nn\endcsname{\let\PY@bf=\textbf\def\PY@tc##1{\textcolor[rgb]{0.00,0.00,1.00}{##1}}}
\expandafter\def\csname PY@tok@ne\endcsname{\let\PY@bf=\textbf\def\PY@tc##1{\textcolor[rgb]{0.82,0.25,0.23}{##1}}}
\expandafter\def\csname PY@tok@nv\endcsname{\def\PY@tc##1{\textcolor[rgb]{0.10,0.09,0.49}{##1}}}
\expandafter\def\csname PY@tok@no\endcsname{\def\PY@tc##1{\textcolor[rgb]{0.53,0.00,0.00}{##1}}}
\expandafter\def\csname PY@tok@nl\endcsname{\def\PY@tc##1{\textcolor[rgb]{0.63,0.63,0.00}{##1}}}
\expandafter\def\csname PY@tok@ni\endcsname{\let\PY@bf=\textbf\def\PY@tc##1{\textcolor[rgb]{0.60,0.60,0.60}{##1}}}
\expandafter\def\csname PY@tok@na\endcsname{\def\PY@tc##1{\textcolor[rgb]{0.49,0.56,0.16}{##1}}}
\expandafter\def\csname PY@tok@nt\endcsname{\let\PY@bf=\textbf\def\PY@tc##1{\textcolor[rgb]{0.00,0.50,0.00}{##1}}}
\expandafter\def\csname PY@tok@nd\endcsname{\def\PY@tc##1{\textcolor[rgb]{0.67,0.13,1.00}{##1}}}
\expandafter\def\csname PY@tok@s\endcsname{\def\PY@tc##1{\textcolor[rgb]{0.73,0.13,0.13}{##1}}}
\expandafter\def\csname PY@tok@sd\endcsname{\let\PY@it=\textit\def\PY@tc##1{\textcolor[rgb]{0.73,0.13,0.13}{##1}}}
\expandafter\def\csname PY@tok@si\endcsname{\let\PY@bf=\textbf\def\PY@tc##1{\textcolor[rgb]{0.73,0.40,0.53}{##1}}}
\expandafter\def\csname PY@tok@se\endcsname{\let\PY@bf=\textbf\def\PY@tc##1{\textcolor[rgb]{0.73,0.40,0.13}{##1}}}
\expandafter\def\csname PY@tok@sr\endcsname{\def\PY@tc##1{\textcolor[rgb]{0.73,0.40,0.53}{##1}}}
\expandafter\def\csname PY@tok@ss\endcsname{\def\PY@tc##1{\textcolor[rgb]{0.10,0.09,0.49}{##1}}}
\expandafter\def\csname PY@tok@sx\endcsname{\def\PY@tc##1{\textcolor[rgb]{0.00,0.50,0.00}{##1}}}
\expandafter\def\csname PY@tok@m\endcsname{\def\PY@tc##1{\textcolor[rgb]{0.40,0.40,0.40}{##1}}}
\expandafter\def\csname PY@tok@gh\endcsname{\let\PY@bf=\textbf\def\PY@tc##1{\textcolor[rgb]{0.00,0.00,0.50}{##1}}}
\expandafter\def\csname PY@tok@gu\endcsname{\let\PY@bf=\textbf\def\PY@tc##1{\textcolor[rgb]{0.50,0.00,0.50}{##1}}}
\expandafter\def\csname PY@tok@gd\endcsname{\def\PY@tc##1{\textcolor[rgb]{0.63,0.00,0.00}{##1}}}
\expandafter\def\csname PY@tok@gi\endcsname{\def\PY@tc##1{\textcolor[rgb]{0.00,0.63,0.00}{##1}}}
\expandafter\def\csname PY@tok@gr\endcsname{\def\PY@tc##1{\textcolor[rgb]{1.00,0.00,0.00}{##1}}}
\expandafter\def\csname PY@tok@ge\endcsname{\let\PY@it=\textit}
\expandafter\def\csname PY@tok@gs\endcsname{\let\PY@bf=\textbf}
\expandafter\def\csname PY@tok@gp\endcsname{\let\PY@bf=\textbf\def\PY@tc##1{\textcolor[rgb]{0.00,0.00,0.50}{##1}}}
\expandafter\def\csname PY@tok@go\endcsname{\def\PY@tc##1{\textcolor[rgb]{0.53,0.53,0.53}{##1}}}
\expandafter\def\csname PY@tok@gt\endcsname{\def\PY@tc##1{\textcolor[rgb]{0.00,0.27,0.87}{##1}}}
\expandafter\def\csname PY@tok@err\endcsname{\def\PY@bc##1{\setlength{\fboxsep}{0pt}\fcolorbox[rgb]{1.00,0.00,0.00}{1,1,1}{\strut ##1}}}
\expandafter\def\csname PY@tok@kc\endcsname{\let\PY@bf=\textbf\def\PY@tc##1{\textcolor[rgb]{0.00,0.50,0.00}{##1}}}
\expandafter\def\csname PY@tok@kd\endcsname{\let\PY@bf=\textbf\def\PY@tc##1{\textcolor[rgb]{0.00,0.50,0.00}{##1}}}
\expandafter\def\csname PY@tok@kn\endcsname{\let\PY@bf=\textbf\def\PY@tc##1{\textcolor[rgb]{0.00,0.50,0.00}{##1}}}
\expandafter\def\csname PY@tok@kr\endcsname{\let\PY@bf=\textbf\def\PY@tc##1{\textcolor[rgb]{0.00,0.50,0.00}{##1}}}
\expandafter\def\csname PY@tok@bp\endcsname{\def\PY@tc##1{\textcolor[rgb]{0.00,0.50,0.00}{##1}}}
\expandafter\def\csname PY@tok@fm\endcsname{\def\PY@tc##1{\textcolor[rgb]{0.00,0.00,1.00}{##1}}}
\expandafter\def\csname PY@tok@vc\endcsname{\def\PY@tc##1{\textcolor[rgb]{0.10,0.09,0.49}{##1}}}
\expandafter\def\csname PY@tok@vg\endcsname{\def\PY@tc##1{\textcolor[rgb]{0.10,0.09,0.49}{##1}}}
\expandafter\def\csname PY@tok@vi\endcsname{\def\PY@tc##1{\textcolor[rgb]{0.10,0.09,0.49}{##1}}}
\expandafter\def\csname PY@tok@vm\endcsname{\def\PY@tc##1{\textcolor[rgb]{0.10,0.09,0.49}{##1}}}
\expandafter\def\csname PY@tok@sa\endcsname{\def\PY@tc##1{\textcolor[rgb]{0.73,0.13,0.13}{##1}}}
\expandafter\def\csname PY@tok@sb\endcsname{\def\PY@tc##1{\textcolor[rgb]{0.73,0.13,0.13}{##1}}}
\expandafter\def\csname PY@tok@sc\endcsname{\def\PY@tc##1{\textcolor[rgb]{0.73,0.13,0.13}{##1}}}
\expandafter\def\csname PY@tok@dl\endcsname{\def\PY@tc##1{\textcolor[rgb]{0.73,0.13,0.13}{##1}}}
\expandafter\def\csname PY@tok@s2\endcsname{\def\PY@tc##1{\textcolor[rgb]{0.73,0.13,0.13}{##1}}}
\expandafter\def\csname PY@tok@sh\endcsname{\def\PY@tc##1{\textcolor[rgb]{0.73,0.13,0.13}{##1}}}
\expandafter\def\csname PY@tok@s1\endcsname{\def\PY@tc##1{\textcolor[rgb]{0.73,0.13,0.13}{##1}}}
\expandafter\def\csname PY@tok@mb\endcsname{\def\PY@tc##1{\textcolor[rgb]{0.40,0.40,0.40}{##1}}}
\expandafter\def\csname PY@tok@mf\endcsname{\def\PY@tc##1{\textcolor[rgb]{0.40,0.40,0.40}{##1}}}
\expandafter\def\csname PY@tok@mh\endcsname{\def\PY@tc##1{\textcolor[rgb]{0.40,0.40,0.40}{##1}}}
\expandafter\def\csname PY@tok@mi\endcsname{\def\PY@tc##1{\textcolor[rgb]{0.40,0.40,0.40}{##1}}}
\expandafter\def\csname PY@tok@il\endcsname{\def\PY@tc##1{\textcolor[rgb]{0.40,0.40,0.40}{##1}}}
\expandafter\def\csname PY@tok@mo\endcsname{\def\PY@tc##1{\textcolor[rgb]{0.40,0.40,0.40}{##1}}}
\expandafter\def\csname PY@tok@ch\endcsname{\let\PY@it=\textit\def\PY@tc##1{\textcolor[rgb]{0.25,0.50,0.50}{##1}}}
\expandafter\def\csname PY@tok@cm\endcsname{\let\PY@it=\textit\def\PY@tc##1{\textcolor[rgb]{0.25,0.50,0.50}{##1}}}
\expandafter\def\csname PY@tok@cpf\endcsname{\let\PY@it=\textit\def\PY@tc##1{\textcolor[rgb]{0.25,0.50,0.50}{##1}}}
\expandafter\def\csname PY@tok@c1\endcsname{\let\PY@it=\textit\def\PY@tc##1{\textcolor[rgb]{0.25,0.50,0.50}{##1}}}
\expandafter\def\csname PY@tok@cs\endcsname{\let\PY@it=\textit\def\PY@tc##1{\textcolor[rgb]{0.25,0.50,0.50}{##1}}}

\def\PYZbs{\char`\\}
\def\PYZus{\char`\_}
\def\PYZob{\char`\{}
\def\PYZcb{\char`\}}
\def\PYZca{\char`\^}
\def\PYZam{\char`\&}
\def\PYZlt{\char`\<}
\def\PYZgt{\char`\>}
\def\PYZsh{\char`\#}
\def\PYZpc{\char`\%}
\def\PYZdl{\char`\$}
\def\PYZhy{\char`\-}
\def\PYZsq{\char`\'}
\def\PYZdq{\char`\"}
\def\PYZti{\char`\~}
% for compatibility with earlier versions
\def\PYZat{@}
\def\PYZlb{[}
\def\PYZrb{]}
\makeatother


    % Exact colors from NB
    \definecolor{incolor}{rgb}{0.0, 0.0, 0.5}
    \definecolor{outcolor}{rgb}{0.545, 0.0, 0.0}



    
    % Prevent overflowing lines due to hard-to-break entities
    \sloppy 
    % Setup hyperref package
    \hypersetup{
      breaklinks=true,  % so long urls are correctly broken across lines
      colorlinks=true,
      urlcolor=urlcolor,
      linkcolor=linkcolor,
      citecolor=citecolor,
      }
    % Slightly bigger margins than the latex defaults
    
    \geometry{verbose,tmargin=1in,bmargin=1in,lmargin=1in,rmargin=1in}
    
    

    \begin{document}
    
    
    \maketitle
    
    

    
    \section{The numpy.random package}\label{the-numpy.random-package}

    \subsection{About Numpy.random}\label{about-numpy.random}

The numpy.random package is a sub-package of the Numpy library{[}1{]}
for the Python programming language. Numpy is used for dealing with
multi-dimensional arrays of values and matrix operations in data
analysis. While it's a powerful and widely used package, users seldom
interact directly with it - it's typically accessed via one of the
sub-packages such as numpy.random.

    \subsubsection{It's Overall Purpose}\label{its-overall-purpose}

The purpose of numpy.random is to provide users with a selection of
methods by which pseudo random numbers can be generated for use in
various applications. A typical application is in the creation of
sampling plans.

While its not currently possible for a computer to create truly random
values, pseudo random numbers (derived from any method using
computational algorithms) offer a useful simulation of actual random
values. In theory, a higher degree of apparent randomness may
characterise numbers generated from methods using electromagnetic
atmospheric noise as a source input. {[}2{]} With the advent of `quantum
computing' perhaps it may finally become possible to generate truly
random values.

    \subsubsection{Who needs random
numbers?}\label{who-needs-random-numbers}

Aside from their frequent use in sampling and simulations, one
real-world application for pseudo random numbers which I have personally
encountered is in the online gaming industry. In this case, a stream of
pseudo random numbers is accessed by connected gaming terminals in
bookmakers shops, pubs etc to produce results in so-called ``games of
chance'', where the user's skills have no influence on the result. As a
mission critical component in the network, the random number generator
(RNG) runs 24x7 for 365 days a year, and any interruption of service is
flagged by loud alarms in the HQ where the operations team quickly work
to restore service!

    \subsubsection{Sampling Methods}\label{sampling-methods}

Sampling Methods can be classified into one of two categories:{[}3{]}

\begin{enumerate}
\def\labelenumi{\arabic{enumi})}
\item
  Probability Sampling: The sample has a known probability of being
  selected. All of the distributions illustrated here are examples of
  probability sampling.These techniques are more likely to yield a
  sample which is representative of the whole population, to within a
  calculated margin of error.
\item
  Non-probability Sampling: The sample does not have a known probability
  of being selected. In contrast with probability sampling, a
  non-probability sample is not a product of a randomized selection
  processes. Subjects in a non-probability sample are usually selected
  on the basis of their accessibility or by the personal judgment of the
  researcher. {[}4{]} Such techniques are unlikely to yield a sample
  which is representative of the whole population.
\end{enumerate}

    \subsection{The use of the Simple Random Data
functions}\label{the-use-of-the-simple-random-data-functions}

The ``Simple random data'' functions{[}5{]} in numpy.random provide a
range of methods for creating random values in a variety of arrays,
formats and structures, depending on the specific needs of the user.
This is a good starting point for devising a basic sampling plan.

    \subsubsection{Using numpy.random.rand}\label{using-numpy.random.rand}

Simple random data functions in numpy.random begin with
numpy.random.rand {[}6{]}, which allows you to create an array of a
given shape and populate it with random floating values of a uniform
distribution, from 0 to 0.99999999. For example, here's an array of 3
sets of sample values, of 4 rows and 3 columns each, generated by
np.random.rand

    \begin{Verbatim}[commandchars=\\\{\}]
{\color{incolor}In [{\color{incolor}1}]:} \PY{k+kn}{import} \PY{n+nn}{numpy} \PY{k}{as} \PY{n+nn}{np}
\end{Verbatim}


    \begin{Verbatim}[commandchars=\\\{\}]
{\color{incolor}In [{\color{incolor}2}]:} \PY{n}{np}\PY{o}{.}\PY{n}{random}\PY{o}{.}\PY{n}{rand}\PY{p}{(}\PY{l+m+mi}{3}\PY{p}{,}\PY{l+m+mi}{4}\PY{p}{,}\PY{l+m+mi}{3}\PY{p}{)}
\end{Verbatim}


\begin{Verbatim}[commandchars=\\\{\}]
{\color{outcolor}Out[{\color{outcolor}2}]:} array([[[0.11584083, 0.55780048, 0.3026395 ],
                [0.57788441, 0.33343881, 0.7080655 ],
                [0.54672904, 0.02515292, 0.8476291 ],
                [0.05932114, 0.58897054, 0.72657292]],
        
               [[0.44632679, 0.56866086, 0.72533905],
                [0.91064284, 0.95655994, 0.44072363],
                [0.65832166, 0.14270782, 0.83314889],
                [0.21844577, 0.12785032, 0.07137863]],
        
               [[0.13937665, 0.77280845, 0.90166358],
                [0.20372351, 0.99773058, 0.64093209],
                [0.40834524, 0.6813582 , 0.73770539],
                [0.37416201, 0.02857268, 0.12634336]]])
\end{Verbatim}
            
    Even in such small sample sizes, each of these values is almost as
likely to appear in the array as any other, as it's a uniform
distribution.

    \subsubsection{Using numpy.random.int}\label{using-numpy.random.int}

    Another simple random data function is numpy.random.int {[}7{]}, which
can create an array of integers between defined low \& high limits, from
the ``discrete uniform'' distribution in the closed interval {[}low,
high{]}. For example, to create an array of 10 integers, greater than 0
and less than or equal to 100 :

    \begin{Verbatim}[commandchars=\\\{\}]
{\color{incolor}In [{\color{incolor}3}]:} \PY{n}{np}\PY{o}{.}\PY{n}{random}\PY{o}{.}\PY{n}{randint}\PY{p}{(}\PY{l+m+mi}{100}\PY{p}{,} \PY{n}{size}\PY{o}{=}\PY{l+m+mi}{10}\PY{p}{)}
\end{Verbatim}


\begin{Verbatim}[commandchars=\\\{\}]
{\color{outcolor}Out[{\color{outcolor}3}]:} array([69, 70, 10, 53, 71, 96, 20, 33, 94, 80])
\end{Verbatim}
            
    \subsection{Permutations}\label{permutations}

The ``Permutation'' functions offer a range of methods for randomising
the arrangement (or order) of a defined sequence of values. This could
be a good starting point for a programmer creating a music shuffle
function.

    \begin{Verbatim}[commandchars=\\\{\}]
{\color{incolor}In [{\color{incolor}4}]:} \PY{k+kn}{import} \PY{n+nn}{numpy} \PY{k}{as} \PY{n+nn}{np}
\end{Verbatim}


    Here is an example of a permutation function using the shuffle command,
starting with an initial array:

    \begin{Verbatim}[commandchars=\\\{\}]
{\color{incolor}In [{\color{incolor}5}]:} \PY{n}{arr} \PY{o}{=} \PY{n}{np}\PY{o}{.}\PY{n}{arange}\PY{p}{(}\PY{l+m+mi}{15}\PY{p}{)}
        \PY{n}{arr}
\end{Verbatim}


\begin{Verbatim}[commandchars=\\\{\}]
{\color{outcolor}Out[{\color{outcolor}5}]:} array([ 0,  1,  2,  3,  4,  5,  6,  7,  8,  9, 10, 11, 12, 13, 14])
\end{Verbatim}
            
    Here it is, randomly shuffled:

    \begin{Verbatim}[commandchars=\\\{\}]
{\color{incolor}In [{\color{incolor}6}]:} \PY{n}{np}\PY{o}{.}\PY{n}{random}\PY{o}{.}\PY{n}{shuffle}\PY{p}{(}\PY{n}{arr}\PY{p}{)}
        \PY{n}{arr}
\end{Verbatim}


\begin{Verbatim}[commandchars=\\\{\}]
{\color{outcolor}Out[{\color{outcolor}6}]:} array([10, 11,  4,  0,  5,  1,  8,  6,  2, 13,  3, 14,  7,  9, 12])
\end{Verbatim}
            
    Incidently, how many permutations are possible in this example? How many
unique ways could you arrange 15 books on a shelf? The answer is
15\emph{14}13\emph{12}11\emph{10}9\emph{8}7\emph{6}5\emph{4}3\emph{2}1
or "factorial 15", expressed as 15! - and the number of permutations is
1.3076744e+12 .... or 1,307,674,368,000

    \subsection{About Probability
Distributions}\label{about-probability-distributions}

    A~probability distribution~is a mathematical function that provides the
probabilities of occurrence of different possible outcomes in
an~experiment.{[}8{]} The concept of the probability distribution and
the random variables which they describe underlies the mathematical
discipline of probability theory, and the science of statistics. There
is spread or variability in almost any value that can be measured in a
population (e.g. height of people, durability of a metal, sales growth,
traffic flow, etc.).

Examples of Probability distributions include the following:

    \subsubsection{1. Uniform Distribution}\label{uniform-distribution}

    \begin{Verbatim}[commandchars=\\\{\}]
{\color{incolor}In [{\color{incolor}7}]:} \PY{k+kn}{import} \PY{n+nn}{matplotlib}\PY{n+nn}{.}\PY{n+nn}{pyplot} \PY{k}{as} \PY{n+nn}{plt}
\end{Verbatim}


    Lets generate 30,000 random values of between 0 and 0.99999999

    \begin{Verbatim}[commandchars=\\\{\}]
{\color{incolor}In [{\color{incolor}8}]:} \PY{n}{x} \PY{o}{=} \PY{n}{np}\PY{o}{.}\PY{n}{random}\PY{o}{.}\PY{n}{rand}\PY{p}{(}\PY{l+m+mi}{30000}\PY{p}{)}
        \PY{n}{x}
\end{Verbatim}


\begin{Verbatim}[commandchars=\\\{\}]
{\color{outcolor}Out[{\color{outcolor}8}]:} array([0.98139379, 0.43907742, 0.98257735, {\ldots}, 0.32088705, 0.27276816,
               0.51148256])
\end{Verbatim}
            
    To illustrate the distribution of the random values generated by the
np.random.rand function, we can use it to generate a sample of values
and then plot them using the matplotlib function.

    \begin{Verbatim}[commandchars=\\\{\}]
{\color{incolor}In [{\color{incolor}9}]:} \PY{c+c1}{\PYZsh{} Plot the distribution of these values in a histogram}
        \PY{n}{plt}\PY{o}{.}\PY{n}{hist}\PY{p}{(}\PY{n}{x}\PY{p}{)}
\end{Verbatim}


\begin{Verbatim}[commandchars=\\\{\}]
{\color{outcolor}Out[{\color{outcolor}9}]:} (array([3020., 2985., 3066., 2885., 3089., 3019., 2903., 2941., 2957.,
                3135.]),
         array([1.77919018e-05, 1.00013680e-01, 2.00009567e-01, 3.00005455e-01,
                4.00001343e-01, 4.99997231e-01, 5.99993119e-01, 6.99989006e-01,
                7.99984894e-01, 8.99980782e-01, 9.99976670e-01]),
         <a list of 10 Patch objects>)
\end{Verbatim}
            
    \begin{center}
    \adjustimage{max size={0.9\linewidth}{0.9\paperheight}}{output_27_1.png}
    \end{center}
    { \hspace*{\fill} \\}
    
    That's what a randomly generated, essentially uniform distribution looks
like. It's uniform insofaras every value in the range defined had an
equal chance of being picked. We see approx 3,000 appearances of each
value.

Another function which generates a uniform distribution is
np.random.uniform :

    \begin{Verbatim}[commandchars=\\\{\}]
{\color{incolor}In [{\color{incolor}10}]:} \PY{c+c1}{\PYZsh{} randomly generate 1000 values from \PYZgt{}\PYZhy{}10 to \PYZlt{}\PYZhy{}9}
         \PY{n}{s} \PY{o}{=} \PY{n}{np}\PY{o}{.}\PY{n}{random}\PY{o}{.}\PY{n}{uniform}\PY{p}{(}\PY{o}{\PYZhy{}}\PY{l+m+mi}{10}\PY{p}{,}\PY{o}{\PYZhy{}}\PY{l+m+mi}{9}\PY{p}{,}\PY{l+m+mi}{1000}\PY{p}{)}
\end{Verbatim}


    the np.random.uniform function seems to show a quite different looking
distribution for this 1000 unit sample, but the larger the sample size
the more uniform it looks.

    \begin{Verbatim}[commandchars=\\\{\}]
{\color{incolor}In [{\color{incolor}11}]:} \PY{n}{plt}\PY{o}{.}\PY{n}{hist}\PY{p}{(}\PY{n}{s}\PY{p}{)}
\end{Verbatim}


\begin{Verbatim}[commandchars=\\\{\}]
{\color{outcolor}Out[{\color{outcolor}11}]:} (array([ 96.,  96.,  89., 103., 109., 108., 108.,  90., 103.,  98.]),
          array([-9.99994761, -9.90000568, -9.80006375, -9.70012182, -9.60017989,
                 -9.50023796, -9.40029603, -9.3003541 , -9.20041217, -9.10047024,
                 -9.00052831]),
          <a list of 10 Patch objects>)
\end{Verbatim}
            
    \begin{center}
    \adjustimage{max size={0.9\linewidth}{0.9\paperheight}}{output_31_1.png}
    \end{center}
    { \hspace*{\fill} \\}
    
    \subsubsection{2. Normal Distribution}\label{normal-distribution}

    A Normal (or Gaussian) distribution is related to real-valued quantities
that grow linearly (e.g. errors, offsets). Its the most common
continuous distribution. We can use the 'numpy.random.normal' function
which generates a normal distribution looking more like the classic
'bell curve'.

    \begin{Verbatim}[commandchars=\\\{\}]
{\color{incolor}In [{\color{incolor}12}]:} \PY{c+c1}{\PYZsh{} randomly generate 1000 values, normally distributed with a mean of 0, in increments of 0.1}
         \PY{n}{y} \PY{o}{=} \PY{n}{np}\PY{o}{.}\PY{n}{random}\PY{o}{.}\PY{n}{normal}\PY{p}{(}\PY{l+m+mi}{0}\PY{p}{,}\PY{l+m+mf}{0.1}\PY{p}{,}\PY{l+m+mi}{1000}\PY{p}{)}
\end{Verbatim}


    \begin{Verbatim}[commandchars=\\\{\}]
{\color{incolor}In [{\color{incolor}13}]:} \PY{c+c1}{\PYZsh{} plot the distribution}
         \PY{n}{plt}\PY{o}{.}\PY{n}{hist}\PY{p}{(}\PY{n}{y}\PY{p}{)}
\end{Verbatim}


\begin{Verbatim}[commandchars=\\\{\}]
{\color{outcolor}Out[{\color{outcolor}13}]:} (array([  4.,  18.,  69., 186., 216., 234., 164.,  78.,  27.,   4.]),
          array([-0.30954091, -0.24816988, -0.18679886, -0.12542783, -0.06405681,
                 -0.00268578,  0.05868525,  0.12005627,  0.1814273 ,  0.24279833,
                  0.30416935]),
          <a list of 10 Patch objects>)
\end{Verbatim}
            
    \begin{center}
    \adjustimage{max size={0.9\linewidth}{0.9\paperheight}}{output_35_1.png}
    \end{center}
    { \hspace*{\fill} \\}
    
    This Normal distribution shows observations clustered around the Mean
value of zero. The chances of values close to zero being picked are
greater than for those at the extremities of the distribution. Typically
95\% of all observations in a Normal distribution fall within +/- 2
Standard Deviations from the Mean.

    Notice that its the 1st digit in np.random.normal (0,0.1,10000) which
defines the MEAN in this distribution - in this case 0. You can make it
any value you choose. Equally, the 2nd digit defines the intervals in
the range.

    \subsubsection{3. Pareto Distribution}\label{pareto-distribution}

    The Pareto distribution seeks to describe quantities which have a
particular property: namely, that a few items account for a lot of it
and a lot of items account for a little of it.

For example, if we think of wealth distribition, a small fraction of the
people (the few richest ones) tend to account for a large fraction of
total income, and a large fraction of the people tend to account for a
small fraction of total income.{[}9{]}

    \begin{Verbatim}[commandchars=\\\{\}]
{\color{incolor}In [{\color{incolor}14}]:} \PY{c+c1}{\PYZsh{} Use the pareto function to draw samples from the distribution:}
         \PY{n}{a}\PY{p}{,} \PY{n}{m} \PY{o}{=} \PY{l+m+mi}{2}\PY{p}{,} \PY{l+m+mf}{3.}  \PY{c+c1}{\PYZsh{} shape and mode}
         \PY{n}{s} \PY{o}{=} \PY{p}{(}\PY{n}{np}\PY{o}{.}\PY{n}{random}\PY{o}{.}\PY{n}{pareto}\PY{p}{(}\PY{n}{a}\PY{p}{,} \PY{l+m+mi}{1000}\PY{p}{)} \PY{o}{+} \PY{l+m+mi}{7}\PY{p}{)} \PY{o}{*} \PY{n}{m}
         
         \PY{c+c1}{\PYZsh{} Display the histogram of the samples, along with the probability density function:}
         \PY{k+kn}{import} \PY{n+nn}{matplotlib}\PY{n+nn}{.}\PY{n+nn}{pyplot} \PY{k}{as} \PY{n+nn}{plt}
         \PY{n}{count}\PY{p}{,} \PY{n}{bins}\PY{p}{,} \PY{n}{\PYZus{}} \PY{o}{=} \PY{n}{plt}\PY{o}{.}\PY{n}{hist}\PY{p}{(}\PY{n}{s}\PY{p}{,} \PY{l+m+mi}{100}\PY{p}{,} \PY{n}{normed}\PY{o}{=}\PY{k+kc}{True}\PY{p}{)}
         \PY{n}{fit} \PY{o}{=} \PY{n}{a}\PY{o}{*}\PY{n}{m}\PY{o}{*}\PY{o}{*}\PY{n}{a} \PY{o}{/} \PY{n}{bins}\PY{o}{*}\PY{o}{*}\PY{p}{(}\PY{n}{a}\PY{o}{+}\PY{l+m+mi}{1}\PY{p}{)}
         \PY{n}{plt}\PY{o}{.}\PY{n}{plot}\PY{p}{(}\PY{n}{bins}\PY{p}{,} \PY{n+nb}{max}\PY{p}{(}\PY{n}{count}\PY{p}{)}\PY{o}{*}\PY{n}{fit}\PY{o}{/}\PY{n+nb}{max}\PY{p}{(}\PY{n}{fit}\PY{p}{)}\PY{p}{,} \PY{n}{linewidth}\PY{o}{=}\PY{l+m+mi}{2}\PY{p}{,} \PY{n}{color}\PY{o}{=}\PY{l+s+s1}{\PYZsq{}}\PY{l+s+s1}{c}\PY{l+s+s1}{\PYZsq{}}\PY{p}{)}
         \PY{n}{plt}\PY{o}{.}\PY{n}{xlabel} \PY{p}{(}\PY{l+s+s1}{\PYZsq{}}\PY{l+s+s1}{Money (m)}\PY{l+s+s1}{\PYZsq{}}\PY{p}{)}
         \PY{n}{plt}\PY{o}{.}\PY{n}{ylabel} \PY{p}{(}\PY{l+s+s1}{\PYZsq{}}\PY{l+s+s1}{Probability P(m)}\PY{l+s+s1}{\PYZsq{}}\PY{p}{)}
         \PY{n}{plt}\PY{o}{.}\PY{n}{show} \PY{p}{(}\PY{p}{)}
\end{Verbatim}


    \begin{Verbatim}[commandchars=\\\{\}]
/anaconda3/lib/python3.6/site-packages/matplotlib/axes/\_axes.py:6462: UserWarning: The 'normed' kwarg is deprecated, and has been replaced by the 'density' kwarg.
  warnings.warn("The 'normed' kwarg is deprecated, and has been "

    \end{Verbatim}

    \begin{center}
    \adjustimage{max size={0.9\linewidth}{0.9\paperheight}}{output_40_1.png}
    \end{center}
    { \hspace*{\fill} \\}
    
    This pareto distribution illustrates the probabilty of a given
individual securing a given share of the wealth in a country (Money(m))
- It shows that its far more probable they will secure a small share,
than it is likely they will get a large share.

    \subsubsection{4. Binomial Distribution}\label{binomial-distribution}

    The binomial distribution model is an important probability model that
is used when there are two possible outcomes (hence "binomial").
{[}10{]}

For example, adults who suffer a myocardial infarction might survive the
heart attack or not, a medical device such as a coronary stent might be
successfully implanted or not. These are some examples of applications
or processes in which the outcome of interest has two possible values
(i.e., it is dichotomous). The two outcomes are often labeled "success"
and "failure" with success indicating the presence of the outcome of
interest.

The binomial distribution model allows us to compute the probability of
observing a specified number of "successes" when the process is repeated
a specific number of times and the outcome for a given patient is either
a success or a failure.

Use of the binomial distribution requires three assumptions:

\begin{enumerate}
\def\labelenumi{\arabic{enumi}.}
\tightlist
\item
  Each replication of the process results in one of two possible
  outcomes (success or failure),
\item
  The probability of success is the same for each replication, and
\item
  The replications are independent. A success in one patient does not
  influence the probability of success in another.
\end{enumerate}

The numpy.random.binomial function is used here to illustrate the
distribution. {[}11{]}

    \begin{Verbatim}[commandchars=\\\{\}]
{\color{incolor}In [{\color{incolor}15}]:} \PY{c+c1}{\PYZsh{} plot the histogram where the number of trials = 10, }
         \PY{c+c1}{\PYZsh{} the probability of each trial = 50\PYZpc{}, tested 1000 times.}
         
         \PY{n}{plt}\PY{o}{.}\PY{n}{hist}\PY{p}{(}\PY{n}{x} \PY{o}{=} \PY{n}{np}\PY{o}{.}\PY{n}{random}\PY{o}{.}\PY{n}{binomial}\PY{p}{(}\PY{l+m+mi}{10}\PY{p}{,} \PY{l+m+mf}{0.5}\PY{p}{,} \PY{l+m+mi}{1000}\PY{p}{)}\PY{p}{)} 
\end{Verbatim}


\begin{Verbatim}[commandchars=\\\{\}]
{\color{outcolor}Out[{\color{outcolor}15}]:} (array([  2.,  12.,  46.,  91., 220., 246., 217., 122.,  34.,  10.]),
          array([ 0.,  1.,  2.,  3.,  4.,  5.,  6.,  7.,  8.,  9., 10.]),
          <a list of 10 Patch objects>)
\end{Verbatim}
            
    \begin{center}
    \adjustimage{max size={0.9\linewidth}{0.9\paperheight}}{output_44_1.png}
    \end{center}
    { \hspace*{\fill} \\}
    
    So, when tested 1,000 times, 5 out of 10 trials conducted gave a
successful result on 250 occasions.

    \subsubsection{5. Exponential
Distribution}\label{exponential-distribution}

    The exponential distribution can be used to model the time between
events in a continuous Poisson process. It is assumed that independent
events occur at a constant rate. {[}12{]}

This distribution has a wide range of applications, including
reliability analysis of products and systems, queuing theory, and Markov
chains. For example, the exponential distribution can be used to model:

\begin{itemize}
\tightlist
\item
  How long it takes for electronic components to fail
\item
  The time interval between customers' arrivals at a terminal
\item
  Service time for customers waiting in line
\item
  The time until default on a payment (credit risk modeling)
\item
  Time until a radioactive nucleus decays
\end{itemize}

    Lets take an example - arrival times to a bank counter are modeled by a
poisson process with a rate of 30 customers per hour (expressed as 0.5
per minute). Using the random.exponential function we can calculate the
probability of any given arrival rate per minute occuring. {[}12{]}
{[}13{]}

    \begin{Verbatim}[commandchars=\\\{\}]
{\color{incolor}In [{\color{incolor}16}]:} \PY{n}{x} \PY{o}{=} \PY{n}{np}\PY{o}{.}\PY{n}{random}\PY{o}{.}\PY{n}{exponential}\PY{p}{(}\PY{l+m+mf}{0.5}\PY{p}{,}\PY{l+m+mi}{1000}\PY{p}{)}
         \PY{n}{plt}\PY{o}{.}\PY{n}{hist}\PY{p}{(}\PY{n}{x}\PY{p}{)}
         \PY{n}{plt}\PY{o}{.}\PY{n}{xlabel} \PY{p}{(}\PY{l+s+s1}{\PYZsq{}}\PY{l+s+s1}{Arrival Rate per minute (x)}\PY{l+s+s1}{\PYZsq{}}\PY{p}{)}
         \PY{n}{plt}\PY{o}{.}\PY{n}{ylabel} \PY{p}{(}\PY{l+s+s1}{\PYZsq{}}\PY{l+s+s1}{Probability P(x)}\PY{l+s+s1}{\PYZsq{}}\PY{p}{)}
         \PY{n}{plt}\PY{o}{.}\PY{n}{show}\PY{p}{(}\PY{p}{)}
         \PY{c+c1}{\PYZsh{} divide P(x) by 1000 to find the probability rate out of 1.}
\end{Verbatim}


    \begin{center}
    \adjustimage{max size={0.9\linewidth}{0.9\paperheight}}{output_49_0.png}
    \end{center}
    { \hspace*{\fill} \\}
    
    this shows the highest probable arrival rate is 0.5 per minute or less.

    \subsection{Using Seeds in Random Number Generation
(RNG)}\label{using-seeds-in-random-number-generation-rng}

    To initiate a random number sequence a 'seed' value is required.
Typically this could be the timestamp, down to the microsecond, from the
processor when the RNG program starts. But the seed could be generated
from any random source - eg. the temperature of the processor, or a
number found somewhere in the decimal expansion of the value for Pi.

The seed is critical to the security of a random number sequence. As
long as it's unknown to any program trying to 'crack' it, it's virtually
impossible to guess what the sequence will be. Equally, if the seed is
known, and tests are run with enough RNG algorithms, then the "random"
sequence will be discovered - a useful reminder that the sequence
generated is only "pseudo random". Using a given seed and a given
algoritm, the same number sequence will be generated every time.

The numpy.random library includes four functions which can be used in
random number generation, examples of which now follow. {[}14{]}
{[}15{]}

    \subsubsection{1.
Numpy.random.RandomState}\label{numpy.random.randomstate}

    RandomState exposes a number of methods for generating random numbers
drawn from a variety of probability distributions. The command
np.random.RandomState(seed=None) constructs a random number generator.
If seed is None, then RandomState will try to read data from
/dev/urandom (or the Windows analogue) if available or seed from the
clock otherwise.

Lets create the RNG and call a random sequence of 5 numbers:

    \begin{Verbatim}[commandchars=\\\{\}]
{\color{incolor}In [{\color{incolor}17}]:} \PY{n}{rng} \PY{o}{=} \PY{n}{np}\PY{o}{.}\PY{n}{random}\PY{o}{.}\PY{n}{RandomState}\PY{p}{(}\PY{n}{seed} \PY{o}{=} \PY{k+kc}{None}\PY{p}{)}
\end{Verbatim}


    \begin{Verbatim}[commandchars=\\\{\}]
{\color{incolor}In [{\color{incolor}18}]:} \PY{n}{rng}\PY{o}{.}\PY{n}{randn}\PY{p}{(}\PY{l+m+mi}{5}\PY{p}{)}
\end{Verbatim}


\begin{Verbatim}[commandchars=\\\{\}]
{\color{outcolor}Out[{\color{outcolor}18}]:} array([-0.26021416, -1.16314182,  0.18769508,  0.49223594,  1.31144861])
\end{Verbatim}
            
    now run it again, with identical calls:

    \begin{Verbatim}[commandchars=\\\{\}]
{\color{incolor}In [{\color{incolor}19}]:} \PY{n}{rng2} \PY{o}{=} \PY{n}{np}\PY{o}{.}\PY{n}{random}\PY{o}{.}\PY{n}{RandomState}\PY{p}{(}\PY{n}{seed} \PY{o}{=} \PY{k+kc}{None}\PY{p}{)}
\end{Verbatim}


    \begin{Verbatim}[commandchars=\\\{\}]
{\color{incolor}In [{\color{incolor}20}]:} \PY{n}{rng2}\PY{o}{.}\PY{n}{randn}\PY{p}{(}\PY{l+m+mi}{5}\PY{p}{)}
\end{Verbatim}


\begin{Verbatim}[commandchars=\\\{\}]
{\color{outcolor}Out[{\color{outcolor}20}]:} array([-1.1232367 ,  1.1895772 ,  0.36698392, -0.07879   , -1.17062428])
\end{Verbatim}
            
    It returned a different number sequence as (seed = None). What if we
define the seed ?

    \begin{Verbatim}[commandchars=\\\{\}]
{\color{incolor}In [{\color{incolor}21}]:} \PY{n}{rng} \PY{o}{=} \PY{n}{np}\PY{o}{.}\PY{n}{random}\PY{o}{.}\PY{n}{RandomState}\PY{p}{(}\PY{n}{seed} \PY{o}{=} \PY{l+m+mi}{123}\PY{p}{)}
\end{Verbatim}


    \begin{Verbatim}[commandchars=\\\{\}]
{\color{incolor}In [{\color{incolor}22}]:} \PY{n}{rng}\PY{o}{.}\PY{n}{randn}\PY{p}{(}\PY{l+m+mi}{5}\PY{p}{)}
\end{Verbatim}


\begin{Verbatim}[commandchars=\\\{\}]
{\color{outcolor}Out[{\color{outcolor}22}]:} array([-1.0856306 ,  0.99734545,  0.2829785 , -1.50629471, -0.57860025])
\end{Verbatim}
            
    run the RNG again, with the same seed, and we get the same array of
values:

    \begin{Verbatim}[commandchars=\\\{\}]
{\color{incolor}In [{\color{incolor}23}]:} \PY{n}{rng3} \PY{o}{=} \PY{n}{np}\PY{o}{.}\PY{n}{random}\PY{o}{.}\PY{n}{RandomState}\PY{p}{(}\PY{n}{seed} \PY{o}{=} \PY{l+m+mi}{123}\PY{p}{)}
         \PY{n}{rng3}\PY{o}{.}\PY{n}{randn}\PY{p}{(}\PY{l+m+mi}{5}\PY{p}{)}
\end{Verbatim}


\begin{Verbatim}[commandchars=\\\{\}]
{\color{outcolor}Out[{\color{outcolor}23}]:} array([-1.0856306 ,  0.99734545,  0.2829785 , -1.50629471, -0.57860025])
\end{Verbatim}
            
    \subsubsection{2. numpy.random.seed}\label{numpy.random.seed}

This method is called when RandomState is initialized (see above). It
can be called again to re-seed the generator. You can also use it to
manually set the seed for randint to generate random numbers.

For example, you could set the seed to 5678 and then ask randint to
return a random integer within a defined range, as follows:

    \begin{Verbatim}[commandchars=\\\{\}]
{\color{incolor}In [{\color{incolor}24}]:} \PY{n}{np}\PY{o}{.}\PY{n}{random}\PY{o}{.}\PY{n}{seed}\PY{p}{(}\PY{l+m+mi}{5678}\PY{p}{)}
         \PY{n}{np}\PY{o}{.}\PY{n}{random}\PY{o}{.}\PY{n}{randint}\PY{p}{(}\PY{l+m+mi}{1}\PY{p}{,} \PY{l+m+mi}{100}\PY{p}{)}
\end{Verbatim}


\begin{Verbatim}[commandchars=\\\{\}]
{\color{outcolor}Out[{\color{outcolor}24}]:} 24
\end{Verbatim}
            
    \subsubsection{3.
Numpy.random.get\_state()}\label{numpy.random.get_state}

This command returns a tuple representing the internal state of the
generator. A tuple is a sequence of values, like a list, except that
tuples are immutable. The tuple returned by get\_state can be used much
like a seed in order to create reproducible sequences of random numbers
{[}16{]}

The returned tuple has the following items: * the string `MT19937'. * a
1-D array of 624 unsigned integer keys. * an integer pos. * an integer
has\_gauss. * a float cached\_gaussian.

    \begin{Verbatim}[commandchars=\\\{\}]
{\color{incolor}In [{\color{incolor}25}]:} \PY{n}{np}\PY{o}{.}\PY{n}{random}\PY{o}{.}\PY{n}{get\PYZus{}state}\PY{p}{(}\PY{p}{)}
\end{Verbatim}


\begin{Verbatim}[commandchars=\\\{\}]
{\color{outcolor}Out[{\color{outcolor}25}]:} ('MT19937', array([2817670114, 1962497881,  914027948, 1463440343, 1938887794,
                 1092405273, 2726328503, 2343148856,  741258043,  273416092,
                 2344723317,  193056740, 3743633300,  957678078, 3876900885,
                 2760083594, 1938161523, 2216869175, 2220072100, 2931112756,
                 1806881730, 2710473768, 2704375294, 3428424545, 2101762119,
                  160709904, 1056457662, 1569491757,  101953039, 1117059617,
                 3810068043, 2107176353, 3128530458,  162517664, 1413004986,
                  134885877, 1866579287,  243054771, 3354272377, 3316485039,
                  766475017, 2886663670, 3827112345, 4015725661, 4012480437,
                 1956829835, 1854065995,  427733846,  798972448, 1127767341,
                 3551560354,  972061951,  511129661, 4186138517, 2219638225,
                  791231207, 1996538239, 1598374079, 3179020503,  778160734,
                  225435774, 4097810135,  912937395,  480971914,  247785298,
                 3532107327,  186534415, 2564184188,  739276050,  249105914,
                  871771753, 1590879840, 3014046134, 2979433626, 1293579635,
                 4028707217, 2340572346, 3834325024,  559537776, 3562682152,
                 2397833977,  128404693,   45140422, 3102295860, 3598054190,
                 2538654404, 3073457860, 1738655095, 2163176771, 3975366148,
                 1789614528, 3517548346, 1678102157,   55323005, 2982654071,
                 3399073617, 2615081231,  219102024, 1751353693,  518525543,
                 2171388466, 3509763601,  678451266, 2886384566, 4062643214,
                 3155917445,   22394711, 2048491440, 3358654631,  778478602,
                  897927217, 3800224196,  734761967, 2154344267, 3164245220,
                 3634430628, 3672126852,  868243820, 3788726784, 1831562131,
                 3414628976, 1889055133, 2748708602, 1160134254,  683947598,
                 2011760320, 3943791367,  954291000, 3964462072, 4238150832,
                 1794304324,  351768995, 2427755322, 1992394311, 1173798343,
                 2912448822,   51540805, 2905420917, 2227016134, 2276595563,
                 3089372840, 3360648235, 2225397815, 1730476677, 2472464821,
                 3749991048, 2587793748,  671082553, 2603864753, 1354625044,
                 1003964990, 2217853131, 3509112685, 3724421683, 1218812061,
                 1015520667, 2008755200, 2689373480, 1910614172,   40171864,
                  226981426, 2074595458, 3610746404, 3449269644, 1783495033,
                 3970438649, 2530219921, 4167451715, 2830960449, 3667835074,
                 2656553762, 3049437587,  506510971, 1802724059, 3205445418,
                  120774861,  285481882, 1289715544, 3464318015, 4058957503,
                 2737261791, 3787497035, 2856298761, 1530200346,  592558550,
                 2692524434, 1036326026, 4200933164, 3431508851, 3669262824,
                  939015566,  246584935,  334237931,  800439876, 2984276990,
                 1088093126,  280357961, 2467022602, 1291680772, 1354871913,
                  559833127, 3767532335,  905451115, 3080218601, 1797900461,
                 3473867913, 3604355709, 1083510274,  371110119, 1652556421,
                 3358075494, 1337830822, 1318416414,  172067910,  991239634,
                 3663618079,  927449845, 3359964920, 1709337624,  878998975,
                 1289056074,  756311813,  689254267, 3849718506, 3992287769,
                 1108889383, 1167567983, 4186616415, 3063288856, 3419258003,
                 3752105691, 3265956040, 3183116078, 3787696341, 1482301283,
                 3287081797, 2313455544, 1800456801, 3914299410, 3190715604,
                 1722674990, 3831978140, 1909890676, 3242833565,  927106087,
                 2180299394, 1523607349,  629676213,  550750616, 1451145908,
                 1997717752, 1775767899, 3284324384, 4092969334, 1811111016,
                 3969386966, 2494013373,  278695081, 3575503596, 3057287430,
                 3406217612, 3155426250, 3496905987, 3473706987, 2291950847,
                 1949357200, 1496556195, 4225375428, 2286861477, 1317166270,
                 1896771549, 1091147743, 3611033253, 1815704021, 3358425549,
                 2351391983, 3694031144,  985524441, 3391904900, 1774151320,
                 2159130726, 1707666615, 3895381021, 3079128276, 1186476218,
                 3010387807, 2493547446,  887169597, 3508460133, 3786296206,
                 1489020761, 1726532414, 3031764857, 2200117555, 4140510988,
                 3921582102, 1510964147, 1779899540, 2833231038, 2978716237,
                  145134040, 1267397115, 1149277041, 3987418548, 2151300886,
                 2811673910, 2372185964, 4127066666, 1139488498,  674027832,
                 3981359069, 3086206730, 2307477831, 3482440227,  154376057,
                 1878452647, 1794437182, 4266058677, 2741697267, 2531332968,
                 3776721477, 3900987432,   36172857,  926763607, 2202740850,
                 4082352684,    1653915, 3445304647, 1632638979,  282717380,
                 3057061760, 3010301271, 1648591364, 2845943783, 3265166365,
                 3974548997,  357325135, 3035326074, 3857405956,    5278341,
                 2497456484, 3872588341, 1367572571,  898412600, 2769871368,
                  732904734,  681539507, 2676283587, 3590495503, 2940344042,
                 2539982549, 3230334122, 2569860493, 3845369932, 3630378244,
                 3304913081, 3159279893, 1093119624, 1071568193,   82497397,
                 1330414704, 2763032493, 1320406096,  489232630,  467455487,
                 3059154310, 3089214913,  234603019, 1415765328,  458653687,
                  753242311, 3081039729, 2487002286, 3506070602, 2148119415,
                 3098106688,  736121099, 2614695755,  833146587, 2769701225,
                 2473883737,  792682347, 1784628606, 1851264025,  373042794,
                 1950298670, 2669020028, 3853968123, 3873071917,  354661617,
                 3902163172, 3390744280,  803066361, 1757874243,  485459997,
                 3750467591, 3324619477, 2715728376, 3899715954,  340694262,
                 4229754221, 1736570486,  642796006, 3131649334, 2156446275,
                 1145826097, 3944203596, 4192099737,  599522907, 1009514492,
                 2017277704, 3522120950,  824970276, 4275928028, 3110508806,
                 2229576116,  879888345, 1069233977, 3784476199, 2074213827,
                 2004482550, 3845940876, 2242267435, 3521622579, 3776870193,
                 2290260658, 1291303324, 1093236285, 4060519652, 1388256226,
                 2442201726, 3092922314, 2601981441, 3891528444, 1871607667,
                 3740376708,  788760987, 4160936919, 1973318853, 2293644281,
                 2082465866, 2600601414,  418585295, 2679699025, 2124830274,
                 1355727264, 1372529968, 3716008808, 1180602362, 3626893421,
                 1528899099, 3447342865,  796386309, 2332170708, 3709851733,
                 1251858986,  319056972,  689572088, 1900764742, 3753660760,
                 3746097394, 2952840491,  407635855,  789780868, 2645166252,
                 3380599825, 4213831858, 1112216582, 2949354838,  671959926,
                 1535515383, 2292957951, 1560628503, 2353651420, 1227083750,
                 2324389144, 4225252098, 1343975588, 1197704373,  738308882,
                 1353843672,  223693641, 3377104767, 1521146510, 1642605967,
                  860050235, 1072953538, 2027610398, 1278398949, 3112343049,
                 3236185457, 2514894151, 3059745724, 3206043743,  139779706,
                 3534634109, 4075061841, 1039064211, 2546213156,  674804031,
                 2885868484,   17437002, 4085075236,  341810583,  692682582,
                 4154986476, 1510748960, 1628082196, 3632959063, 3354355399,
                 2723615584,  645908691, 3772690916, 2388279422, 3217347595,
                 1361320475, 2552340315, 2814125791, 2890394458,  286084571,
                 1322489372,  880181023, 2987438991,  675082317, 2082599547,
                 2932637932,  503674355, 1325961264, 2458465233, 2989527121,
                 1485366991, 2213852072, 1234612876, 2982625160, 3836737944,
                  491767641, 2905650532, 1267984236,   25501094, 2574637610,
                  749382742, 1938070271, 1389639687, 3249756419, 1551071824,
                 2444080128, 2188109259, 3663312932, 2963596685, 1808032510,
                  751995678, 2857204627, 2698411787, 2048904072, 3621406208,
                 2067352346, 2472834792,  186744311, 2123039564, 3854458821,
                 4256681441, 3676403562,  139842173,  575481010, 3271635011,
                 1774756753,   37330565, 2952055928, 3847563246, 1253024735,
                 2894309590,  728893682,  247275124, 2324873167,   27311218,
                  610390349, 1380221081, 3228030647, 1159437758, 1146806740,
                 3968062719, 2302398553,  683833865, 4096658847, 4144921537,
                  964333074, 2334002254, 3936007779,  115667152, 1366145601,
                 3728903518, 4273196616,  882523241,   38666453, 2413696916,
                 4088501651, 3406145203, 3626039687, 2018653487,  165529522,
                 3122612003,  524152036, 2610463132, 3425863109,  385626999,
                 2386657396, 1433675122, 2480768733, 4225660701, 3710485553,
                 3939001287, 3298713169, 1407767031,  940603425, 1945276249,
                 1066422304,  245931312, 3473968062, 2720869655, 4273229064,
                  646927159, 1801403261, 2381048872, 2513483556], dtype=uint32), 2, 0, 0.0)
\end{Verbatim}
            
    \subsubsection{4. numpy.random.set\_state}\label{numpy.random.set_state}

    This command sets the internal state of the generator from a tuple (of
the same parameters as that generated by the numpy.random.get\_state()
command (above). It can be used if needed to manually (re-)set the
internal state of the ``Mersenne Twister''{[}R523523{]} pseudo-random
number generating algorithm. {[}17{]}

The command is: numpy.random.set\_state(state) where state : tuple(str,
ndarray of 624 uints, int, int, float)

    \paragraph{The random.uniform
function}\label{the-random.uniform-function}

A range of probability distributions is available to use within the
RandomState function. For a uniform distribution, in which every value
in the range defined has an equal chance of being picked, we can use the
numpy.random.uniform function:

    \begin{Verbatim}[commandchars=\\\{\}]
{\color{incolor}In [{\color{incolor}26}]:} \PY{k+kn}{import} \PY{n+nn}{numpy} \PY{k}{as} \PY{n+nn}{np}
\end{Verbatim}


    \begin{Verbatim}[commandchars=\\\{\}]
{\color{incolor}In [{\color{incolor}27}]:} \PY{c+c1}{\PYZsh{} The sample size is 1000 and all values are within the given interval of \PYZgt{}= \PYZhy{}1 and \PYZlt{}99.}
         \PY{n}{s} \PY{o}{=} \PY{n}{np}\PY{o}{.}\PY{n}{random}\PY{o}{.}\PY{n}{uniform}\PY{p}{(}\PY{o}{\PYZhy{}}\PY{l+m+mi}{1}\PY{p}{,}\PY{l+m+mi}{99}\PY{p}{,}\PY{l+m+mi}{1000}\PY{p}{)}
         \PY{n}{plt}\PY{o}{.}\PY{n}{hist}\PY{p}{(}\PY{n}{s}\PY{p}{)}
\end{Verbatim}


\begin{Verbatim}[commandchars=\\\{\}]
{\color{outcolor}Out[{\color{outcolor}27}]:} (array([107., 109.,  89., 103.,  89., 105.,  99.,  87., 109., 103.]),
          array([-0.9939068 ,  9.00425749, 19.00242177, 29.00058605, 38.99875034,
                 48.99691462, 58.99507891, 68.99324319, 78.99140748, 88.98957176,
                 98.98773604]),
          <a list of 10 Patch objects>)
\end{Verbatim}
            
    \begin{center}
    \adjustimage{max size={0.9\linewidth}{0.9\paperheight}}{output_73_1.png}
    \end{center}
    { \hspace*{\fill} \\}
    
    \paragraph{The random.poisson
function}\label{the-random.poisson-function}

    The Poisson distribution is specified by one parameter: lambda (λ). This
parameter equals the mean and variance. As lambda increases to
sufficiently large values, the normal distribution (λ, λ) may be used to
approximate the Poisson distribution. {[}18{]}

The Poisson distribution can be used to describe the number of times an
event occurs in a finite observation space. For example, a Poisson
distribution can describe the number of defects in the mechanical system
of an airplane or the number of calls to a call center in an hour. The
Poisson distribution is often used in quality control,
reliability/survival studies, and insurance. A variable follows a
Poisson distribution if the following conditions are met:

\begin{itemize}
\tightlist
\item
  Data are counts of events (nonnegative integers with no upper bound).
\item
  All events are independent.
\item
  Average rate does not change over the period of interest.
\end{itemize}

    \begin{Verbatim}[commandchars=\\\{\}]
{\color{incolor}In [{\color{incolor}28}]:} \PY{c+c1}{\PYZsh{} generate two random samples of 100 values each, using a poision distribution with lamdas }
         \PY{c+c1}{\PYZsh{} of 10 and 3.}
         \PY{n}{s} \PY{o}{=} \PY{n}{np}\PY{o}{.}\PY{n}{random}\PY{o}{.}\PY{n}{poisson}\PY{p}{(}\PY{n}{lam}\PY{o}{=}\PY{p}{(}\PY{l+m+mi}{10}\PY{p}{,} \PY{l+m+mi}{3}\PY{p}{)}\PY{p}{,} \PY{n}{size}\PY{o}{=}\PY{p}{(}\PY{l+m+mi}{100}\PY{p}{,} \PY{l+m+mi}{2}\PY{p}{)}\PY{p}{)}
         \PY{n}{plt}\PY{o}{.}\PY{n}{hist}\PY{p}{(}\PY{n}{s}\PY{p}{)}
\end{Verbatim}


\begin{Verbatim}[commandchars=\\\{\}]
{\color{outcolor}Out[{\color{outcolor}28}]:} ([array([ 0.,  1.,  9., 22., 19., 29., 12.,  6.,  1.,  1.]),
           array([41., 39., 20.,  0.,  0.,  0.,  0.,  0.,  0.,  0.])],
          array([ 0. ,  2.1,  4.2,  6.3,  8.4, 10.5, 12.6, 14.7, 16.8, 18.9, 21. ]),
          <a list of 2 Lists of Patches objects>)
\end{Verbatim}
            
    \begin{center}
    \adjustimage{max size={0.9\linewidth}{0.9\paperheight}}{output_76_1.png}
    \end{center}
    { \hspace*{\fill} \\}
    
    \subsubsection{References}\label{references}

    \begin{enumerate}
\def\labelenumi{\arabic{enumi}.}
\tightlist
\item
  http://www.numpy.org/
\item
  https://en.wikipedia.org/wiki/Random\_number\_generation
\item
  https://onlinecourses.science.psu.edu/stat100/node/18/
\item
  https://explorable.com/non-probability-sampling
\item
  https://docs.scipy.org/doc/numpy-1.14.1/reference/routines.random.html\#
\item
  (https://docs.scipy.org/doc/numpy-1.15.1/reference/generated/numpy.random.rand.html\#numpy.random.rand)
\item
  (https://docs.scipy.org/doc/numpy-1.14.1/reference/generated/numpy.random.randint.html\#numpy.random.randint)
\item
  (https://en.wikipedia.org/wiki/Probability\_distribution\#Applications)
\item
  https://math.stackexchange.com/questions/24204/understanding-the-pareto-distribution-as-applied-to-wealth
\item
  http://sphweb.bumc.bu.edu/otlt/MPH-Modules/BS/BS704\_Probability/BS704\_Probability7.html
\item
  https://docs.scipy.org/doc/numpy-1.14.1/reference/generated/numpy.random.binomial.html\#numpy.random.binomial
\item
  https://support.minitab.com/en-us/minitab-express/1/help-and-how-to/basic-statistics/probability-distributions/supporting-topics/distributions/exponential-distribution/
\item
  https://en.wikipedia.org/wiki/Exponential\_distribution
  https://www.youtube.com/watch?v=4PEX-SuftjQ
\item
  https://www.quora.com/What-is-seed-in-random-number-generation
\item
  https://stackoverflow.com/questions/22994423/difference-between-np-random-seed-and-np-random-randomstate
\item
  https://docs.scipy.org/doc/numpy-1.14.1/reference/generated/numpy.random.get\_state.html\#numpy.random.get\_state
\item
  https://docs.scipy.org/doc/numpy-1.14.1/reference/generated/numpy.random.set\_state.html\#numpy.random.set\_state
\item
  https://support.minitab.com/en-us/minitab-express/1/help-and-how-to/basic-statistics/probability-distributions/supporting-topics/distributions/poisson-distribution/
\end{enumerate}


    % Add a bibliography block to the postdoc
    
    
    
    \end{document}
